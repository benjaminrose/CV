% LaTeX resume using res.cls
%changed line from sharelatex.com on the mac. Ipad does not work well.
\documentclass[margin]{res}
% \documentclass{res}
%\usepackage{charter}
\renewcommand*\sfdefault{uop}
%\renewcommand*\familydefault{\sfdefault} %% Only if the base font of the document is to be sans serif
\usepackage[T1]{fontenc}
%\usepackage{helvetica} % uses helvetica postscript font (download helvetica.sty)
%\usepackage{newcent}   % uses new century schoolbook postscript font 
% \setlength{\textwidth}{5.2in}%{5.2in} % set width of text portion
\usepackage{hyperref}% add hypertext capabilities
% \hypersetup{colorlinks=true}
\urlstyle{same}
\usepackage{verbatim} %lets me comment out blocks of text with \begin{comment} and \end{comment}
\usepackage{cleveref}
\usepackage{graphicx}
\usepackage{etaremune}  %reverse counting lists, http://texblog.org/2011/11/29/reverse-enumerate-or-etaremune/
\usepackage[ampersand]{easylist}
% \usepackage{blindtext}
\usepackage{enumitem}


%This lets me add headers/footers
\usepackage{fancyhdr}
\pagestyle{fancy}
\fancyhf{}
% \thispagestyle{\rhead{\includegraphics[scale= 0.18]{BenRose_ND_Contact.png}}}

\addtolength{\oddsidemargin}{-.25in}
\addtolength{\evensidemargin}{-.25in}
\addtolength{\textwidth}{0.5in}
\addtolength{\resumewidth}{0.5in}
% \setlength{\parindent}{15pt}

\renewcommand{\headrulewidth}{0.0pt}
%\lhead{}
%\chead{}
%\rhead{Benjamin Rose CV \thepage} %\thepage} this adds a page number
%\rhead{\includegraphics[scale= 0.15]{BenRose_ND_Contact.png}}
\lfoot{last updated: \today}
\cfoot{}
\rfoot{Benjamin M. Rose CV, page \thepage}
%\rhead{\includegraphics[scale= 0.18]{BenRose_ND_Contact.png}}


\thispagestyle{fancy}
\thispagestyle{empty}
%\rhead{\includegraphics[scale= 0.18]{BenRose_ND_Contact.png}}

\makeatletter
\newcommand\entry{\@startsection{subsubsection}{3}{\z@}%
                                     {-3.25ex\@plus -1ex \@minus -.2ex}%
                                     {-1.5ex \@plus -.2ex}% Formerly 1.5ex \@plus .2ex
                                     {\normalfont\normalsize\bfseries}}
\makeatother



\begin{document}
% Center the name over the entire width of resume:
%\moveleft.5\hoffset\centerline{\hfill \includegraphics[scale =0.17]{BenRose_ND_Contact.png}}
\moveleft.5\hoffset\centerline{\Huge\bf \href{http://www.nd.edu/~brose3}{Benjamin M. Rose}}
% \moveleft.5\hoffset\centerline{\large \bf Physics Ph.D. Candidate at Notre Dame}
\moveleft.5\hoffset\centerline{\bf \href{http://orcid.org/0000-0002-1873-8973}{OrcidID: 0000-0002-1873-8973}}
% Draw a horizontal line the whole width of resume:
\setlength{}{}
\moveleft\hoffset\vbox{\hrule width\resumewidth height 1pt}\smallskip
% address begins here
% Again, the address lines must be centered over entire width of resume:
\moveleft.5\hoffset\centerline{Department of Physics \hfill 574.387.3453}
\moveleft.5\hoffset\centerline{225 Nieuwland Science Hall \hfill \href{mailto:brose3@nd.edu}{\url{brose3@nd.edu}}}
\moveleft.5\hoffset\centerline{Notre Dame, Indiana 46556 \hfill }%\href{http://www.nd.edu/~brose3}{www.nd.edu/\~{}brose3}}
%\moveleft.5\hoffset\centerline{574.387.3453 -- \url{brose3@nd.edu}}
%\moveleft.5\hoffset\centerline{\url{www.nd.edu/~brose3}}
%\moveleft.5\hoffset\centerline{brose3@nd.edu} %try to make clickable using /url??
%\moveleft.5\hoffset\centerline{brose12@my.whitworth.edu}







\begin{resume}

\section{Education}
%\subsubsection{Current Program:} 
% {\bf Current Program:} \\
% {\sl Doctor of Philosophy  \hfill  Expected Graduation 2018} \\
% Advisor: \href{www.nd.edu/~pgarnav}{Peter Garnavich}
% The University of Notre Dame \\
% South Bend, IN

{\bf Master of Science in Physics} \hfill  {\bf 2016}\\
% {\it title} \\
\href{http://physics.nd.edu}{University of Notre Dame}, Notre Dame, Indiana \\ 
Advisor: \href{www.nd.edu/~pgarnavi}{Professor Peter Garnavich}


% {\bf M.S. in Physics\hfill  May 2015 }\\
% \href{http://physics.nd.edu}{University of Notre Dame}, Notre Dame, IN \\ 
% Advisor: \href{www.nd.edu/~pgarnavi}{Peter Garnavich}

% {\bf Completed Education:} \\
{\bf Bachelor of Science in Physics,} \textit{cum laude} \hfill \textbf{2012}\\
% cum laude\\%, GPA: 3.6/4.0 \\
\href{http://www.whitworth.edu/physics/}{Whitworth University}, Spokane, Washington \\ 
Minor: Mathematics \vspace{-2pt}
 

% \section{\href{http://www.nd.edu/~brose3/current-research.html}{Research \\Interests}}
% % \begin{tabular}{l l}
% % Type Ia Supernovae & Observational Cosmology  \\
% % Type Ia Supernovae Host Environment & Nuclear Astrophysics \\
% % Public and Community Based Science
% % \end{tabular}

% \begin{itemize}\itemsep -2pt
% \item Type Ia supernovae%, specifically host environments
% \item Observational cosmology 
% \item Public and open based science
% \end{itemize} 



% \section{Employment}
% \href{http://physics.nd.edu}{\bf University of Notre Dame} {\bf \hfill August 2012 - Present}\\
% {\sl Research Assistant} \hfill 2 semester \\
% {\sl \href{labTA}{Teaching Assistant} \cref*{SI} } \hfill 3 semesters
% %todo(make ref to different section in this text)


% \href{http://www.whitworth.edu/physics/}{\bf Whitworth University} {\bf \hfill February 2011 - May 2012}\\
% {\sl Supplemental Instruction Leader \cref{SI} for Physics } \hfill 2 semesters \\
% {\sl Laboratory Teaching Assistant } \hfill 1 semester
% % \begin{itemize}\itemsep -2pt
% % \item More details under Teaching Experience
% % \end{itemize} 
 
% % {\bf Personal Tutor \hfill September 2011 - May 2012}\\
% % {\sl High School Geometry and Algebra } 

% \href{http://www.tunl.duke.edu}{\bf Triangle Universities Nuclear Laboratory} {\bf \hfill Summer 2011}\\
% {\sl Research Experience for Undergraduates }
% \begin{itemize}\itemsep -2pt
% \item Durham, NC
% % \item Worked on \textit{Determining the Location of a Radioactive Source in \textsc{Majorana Demonstrator}} described in Lab Work \& Projects 
% \end{itemize} 






%this should just be on the web, it seems odd in a CV.
% \section{Research \\Experience}
% {\bf Searching for a Cosmic Bulk Flow} 
% \begin{itemize}\itemsep -2pt
%      \item Minimized around a cosine distribution to look for a directional dependence to the residuals of the Hubble Diagram
%      \item Developed in Python with \href{http://www.astropy.org}{astropy}, \href{http://roban.github.io/CosmoloPy/}{CosmoloPy}, and \href{https://code.google.com/p/pyminuit/}{PyMinuit}
% \end{itemize}

% {\bf Determining the Location of a Radioactive Source in \textsc{Majorana Demonstrator}}
% \begin{itemize}\itemsep -2pt
%    \item Worked at \href{http://www.tunl.duke.edu}{TUNL} analyzing simulated data with \href{http://root.cern.ch/drupal/content/pyroot}{pyROOT} to determine how well the \textsc{Majorana} detector array can resolve the location of a radioactive hot spot.
% \end{itemize}






\section{Observational Experience}
{\href{http://vaticanobservatory.org/VATT/}{\bf  Vatican Advanced Technology Telescope (VATT)}} \hfill \textbf{June 2014} \\
\href{http://mgio.arizona.edu}{Mount Graham International Observatory}, Safford, Arizona \\
% June 2014, 
4 nights
% \begin{itemize}\itemsep -2pt
%   \item[] June 27, 2014 - July 1, 2014
% \end{itemize}

{\href{http://www.sdss.org/surveys/manga/}{\bf SDSS MaNGA}} \hfill \textbf{2017} \\
\href{http://www.sdss.org/dr14/manga/manga-target-selection/ancillary-targets/}{40 ancillary targets}









\section{Professional Societies}
% \begin{tabbing}
% http://www.emerson.emory.edu/services/latex/latex2e/latex2e_61.html
% \hspace{\middlewidth}\= \kill %set up one tab stop, The edge of the 
% \hspace{-1.5em}\begin{tabular}{p{4.17in} r}
{\bf American Astronomical Society (AAS)}, Junior Member  \hfill {\bf 2014 - present}\\
% \begin{itemize}\itemsep -2pt
    % \item[] \textit{\small{Junior Member (2014-?)}}
    % \end{itemize} \vspace{-10pt}
% \textbf{\href{http://network.asa3.org}{American Scientific Affiliation (ASA)}}, Student Member \hfill \textbf{2014 - present}\\
\textbf{\href{gradphysics.nd.edu}{Graduate Physics Society (GPS)}}%, Member \hfill {\bf 2012 - present} \\
% $_{•}    $~~~~Executive Board Member \hfill {\bf 2015 - 2017}\\
\begin{itemize}\itemsep -2pt
    \item[] Member \hfill {\bf 2012 - present}
    \item[] Puplic Relations Chair \hfill {\bf 2017 - present}
    \item[] Executive Board Member \hfill {\bf 2015 - 2017}
    \item[] Annual Conference co-chair \hfill {\bf 2016}
\end{itemize} \vspace{-12pt}
{\bf American Physical Society (APS)}
\begin{itemize}\itemsep -2pt
    \item[] Student Member  \hfill {\bf 2011 - 2014}
\end{itemize} \vspace{-2pt}
% Student Member
% \end{tabular}
% \end{tabbing}




\section{Activities \& Outreach} %Service and outreach??

% \hspace{-1.5em}\begin{tabular*}{\resumewidth}{p{4.17in} r}
% \begin{description}[align=left] % use \itme[-]
% \ListProperties(Style2*=,Numbers=a,Numbers1=R,FinalMark={)})
{\bf \href{gradphysics.nd.edu}{Graduate Physics Society (GPS) PR Chair}} \hfill {\bf 2017 - present}\\
{\bf \href{gradphysics.nd.edu}{Graduate Physics Society (GPS) Executive Board Member}} \hfill {\bf 2015 - 2017}
\\
    % \begin{itemize}\itemsep -2pt
    % \item[] \textit{\small{Executive Board Member promoting the Socieity's goal of ``fostering a \\community built on intellectual, professional, and social interactions.''}}
    % \end{itemize} \vspace{-12pt}
{\bf \href{http://gradphysics.nd.edu/conference/gpsac-2016/}{Graduate Physics Society Annual Conference Co-chair}} \hfill {\bf 2016 }
% \\Member of the Oranizing Committee\\
    \begin{itemize}\itemsep -2pt
    \item[] \textit{\small{We organized a conference with over 60 attendees.}} % A total of 16 talks and posters
    \end{itemize} \vspace{-12pt}
%\href{http://gradphysics.nd.edu/2016/11/30/gpsac-2016-agenda/}{The conference had over 10 posters \\and 6 talks.}
    % \begin{itemize}\itemsep -2pt
    % \item[] Organizing Committee Member 
    % \end{itemize} \vspace{-10pt}   
% {\bf American Astronomical Society (AAS)}, Junior Member  \hfill {\bf 2014 - present }\\
{\bf Graduate Student Union (GSU)} \hfill {\bf 2013 - 2014}
%\\ \textit{\small{I was the Physics Department representative and worked on issues of parking, \\health insurance, building remodels, and more.}}\\
    \begin{itemize}\itemsep -2pt
    \item[] \textit{\small{I was the Physics Department representative and worked on issues regarding \\parking, health insurance, building remodels and more.}}
    \end{itemize} \vspace{-12pt}
% {\bf American Physical Society (APS)}, Student Member  \hfill {\bf 2011 - 2014 }\\
% {\bf Notre Dame Summer Band} \hfill {\bf 2013 \&  2014 }\\
% {\bf Whitworth University Wind Symphony}  \hfill {\bf 2009 - 2012 }\\
{\bf\href{http://www.whitworthnearspace.org/wiki/Main_Page}{Whitworth Near Space}} \hfill  \textbf{Spring 2012} 
% \\I worked with middle school students on high altitude balloon \\experiments.
% We developed and built radiation, ozone, and temperature \\detector systems. \\
    \begin{itemize}\itemsep -2pt
    % \item[] Worked with middle school students on high altitude balloon experiments
    \item[] \textit{\small{I worked with middle school students on two high altitude balloon experiments. \\We developed and built radiation, ozone, and temperature detector systems.}}
    % \item[] Developed \& built radiation, ozone, and temperature detector systems
    \end{itemize}\vspace{-12}
    
{\bf Club Treasurer}, Whitworth University \hfill {\bf2009 - 2011}
% {\bf Massah} \hfill Summer 2010
% \begin{itemize}  \itemsep -2pt %reduce space between items
%      \item A 10 week program of developing and studying group dynamics and cross-cultural activities in Israel and India
% \end{itemize}
% {\bf Small Group Leader}, Whitworth University \hfill 2009 - 2010
% \begin{itemize}  \itemsep -2pt %reduce space between items
%      \item Co-leading of a student Bible study
% \end{itemize}

% \end{tabular*}
% \end{description}






\section{Awards}

{\bf Lennox Graduate Fellowship}, Notre Dame \hfill {\bf 2017}
\begin{itemize}  \itemsep -2pt %reduce space between items
     \item[] \textit{\small{To recognize achievements and promise as a graduate student in physics}}
\end{itemize} \vspace{-12pt}
{\bf GSU Conference Presentation Grant}, Notre Dame  \hfill {\bf 2015 \& 2016}\\
{\bf Notebaert Professional Development Award}, Notre Dame \hfill {\bf 2015 \& 2016}\\
{\bf Poster Grant}, GSU 6th Annual Research Symposium \hfill {\bf 2014}\\
{\bf Presidential Scholarship}, Whitworth University \hfill {\bf2008 - 2012}\\
{\bf Delbert E. Friesen Memorial Scholarship}, Whitworth University \hfill {\bf2011 - 2012}\\
{\bf Talent Scholarship in Physics}, Whitworth University \hfill {\bf2008 - 2011}
% \\
% {\bf Talent Scholarship in Music}, Whitworth University \hfill {\bf2009 - 2012}\\
% {\bf Laureate Society}, Whitworth University \hfill {\bf4 semesters}
% %\\For a semester GPA of 3.75 or greater
% \begin{itemize}  \itemsep -2pt %reduce space between items
%      \item[] \textit{\small{For a semester GPA of 3.75 or greater}}
% \end{itemize} %\vspace{-12pt}








\section{\href{https://github.com/benjaminrose}{Computer \\Skills}}
%{\bf Computer Languages \& Programs:}
 \hspace{-1em}
 \begin{tabular}{l p{4.2in}}
{\bf Daily Use:} & \href{https://www.python.org}{Python}, \href{https://git-scm.com}{git}, \href{https://github.com/benjaminrose}{GitHub}, \href{http://www.astropy.org}{Astropy}, \LaTeX, \href{http://daringfireball.net/projects/markdown/}{Markdown}, macOS, numpy, scipy\\
{\bf Proficient:} &  \href{http://jupyter.org}{Jupyter Notebook}, \href{https://jekyllrb.com}{Jekyll}, HTML, CSS, Wordpress, TravisCI, pytest, codcov, \href{http://dan.iel.fm/emcee/current/}{emcee}\\
{\bf Competent:} & \href{http://pandas.pydata.org}{pandas}, Linux, Windows %\\
%{\bf Some Experience:} & C++, Parallel Computing, Mathematica, MATLAB, Javascript, Apple Script, ROOT, Swift\\
\end{tabular}
% {\bf Some Experience:} LabVIEW, Apple Script, Julia, Parallel Computing, Swift

\section{\href{https://github.com/benjaminrose}{Open Source Contributions}}
\hspace{-1em}
\begin{tabular}{l p{4.9in}}
\href{http://sep.readthedocs.io/en/v1.0.x/}{\texttt{sep}} & \href{https://github.com/kbarbary/sep/commit/612033788bcce44f110a87e1b54bb70eea9960c2}{accepted documentation update} \\

\href{https://www.scipy.org}{\texttt{scipy}} & submitted a documentation update of \href{https://github.com/scipy/scipy/issues/7204}{limitation in \texttt{integrate.quad}} \\

\href{http://dan.iel.fm/emcee/current/}{\texttt{emcee}} & \href{https://github.com/dfm/emcee/pull/212}{submitted multiple documentation update} \\

\href{http://www.astropy.org}{\texttt{astropy}} & \href{https://github.com/astropy/astropy/issues/4976}{found issue in world coordinate system utility} 
% \\

% \href{https://github.com/astrofrog/acknowledgment-generator}{Acknowledgment Generator} & \href{https://github.com/astrofrog/acknowledgment-generator/pull/63}{Added SDSS IV}
\end{tabular}





\section{Teaching Experience}\label{teaching}
\textbf{Intro. to Scientific Computing with Python} \hfill{} \textbf{Summer 2017}
\begin{itemize}\itemsep -2pt
    \item[] \textit{\small{Lead instructor for introduction to python and computational methods \\course for the Notre Dame REU program}}
    \end{itemize} \vspace{-12pt}
\textbf{Physics GRE Preparation Course} \hfill{} \textbf{Summer 2017}
\begin{itemize}\itemsep -2pt
    \item[] \textit{\small{Solo instructor for a review course of the material of the Physics GRE \\exam for the Notre Dame REU program}}
    \end{itemize} \vspace{-12pt}
% \textbf{Individual Tutoring} \hfill{} \textbf{Fall 2015 \& 2016}
% % \\Sessions included topics in math methods, thermodynamics, and E\&M.\\
%     \begin{itemize}\itemsep -2pt
%     \item[] \textit{\small{Sessions included topics in math methods, thermodynamics, and E\&M.}}
%     \end{itemize} \vspace{-12pt}
\textbf{Intro. to Scientific Computing with Python} \hfill{} \textbf{Spring 2016}
% \\I taught weekly basic programing help session and graded programing\\ assignments.\\
    \begin{itemize}\itemsep -2pt
    \item[] \textit{\small{Taught basic programming help session and graded assignments}}
    \end{itemize} \vspace{-12pt}
% \textbf{various exam grading}
% \textbf{Homework grading} \hfill{} \textbf{A lot}
%     \begin{itemize}\itemsep -2pt
%     \item[] Intro. to Astronomy, Calculus based
%     \item[] Intro. to Astronomy, no math (maybe Intro. to Cosmology)
%     \end{itemize} \vspace{-10pt}
\textbf{Engineering Intro. Physics labs} \hfill{} \textbf{Spring \& Fall 2013}\\
\textbf{Pre-Med. Intro. Physics labs} \hfill{} \textbf{Fall 2012 \& Summer 2013}\\
% $_{•}    $ \hfill \textbf{Summer 2013}
    % \begin{itemize}\itemsep -2pt
    % \item[] Three semesters, multiple sections
    % \end{itemize} \vspace{-10pt}
{\bf Supplemental Instruction Leader} \hfill{} \textbf{Fall 2011 \& Spring 2012}
    \begin{itemize}\itemsep -2pt
    \item[] \textit{\small{Led a group study session of introductory physics material with a focus\\on active learning techniques.}}
    \end{itemize}

% {\bf Personal Tutor:}
% Worked individually with a student on material and study skills for High School Geometry and Algebra.
\begin{comment}
year |   Fall        | Spring        | Summer
1st  | Labs-premed   | labs-eng      | labs-premed
2nd  | Labs-eng      | RA            | RA
3rd  | RA/IntroAstro | Peter's intro | RA
4th  | IntroAstro    | Python        | RA
5th  | Physics C HW  | RA            | RA & REU classes
6th  | Fellowship (GRE Class) | Python | NA!?!
\end{comment}








\section{Oral \\Presentations}
{\sl Searching For a Cosmic-scale Dark Flow} \hfill {\bf November 20, 2015}\\
2015 APS Prairie Section Meeting, Notre Dame

{\sl Finding A Cosmic Bulk Flow} \hfill {\bf April 28, 2014}\\
2014 GPS Spring Conference, Notre Dame 

{\sl Determining the Location of a Radioactive Source in \textsc{Majorana}} \hfill {\bf August 2, 2011} \\
{\sl\textsc{Demonstrator}} \\
REU Culminating Talks,
Duke University %\hfill {\bf August 2, 2011} 


%  $_{•}    $\\\centerline{\bf Talks:} \\ \\
% {\bf Searching For a Cosmic-scale Dark Flow \hfill November 20, 2015} \\
% {\sl 2015 APS Prairie Section Meeting} \\
% The University of Notre Dame \\
% South Bend, IN

% {\bf Finding A Cosmic Bulk Flow \hfill April 28, 2014} \\
% {\sl 2014 GPS Spring Conference} \\
% The University of Notre Dame \\
% South Bend, IN

% {\sl Galaxies talk 3}  \\
% The University of Notre Dame \\
% South Bend, IN
% \begin{itemize}\itemsep -2pt
%      \item A class presentation given as an author on a significant paper from the literature
% \end{itemize}

% {\sl Galaxies talk 2}  \\
% The University of Notre Dame \\
% South Bend, IN
% \begin{itemize}\itemsep -2pt
%      \item A class presentation given as an author on a significant paper from the literature
% \end{itemize}

% {\bf Buldeless Dwarf Galaxies, Presenting on Govenato et al. 2010 \\
%  $_{•}    $ \hfill February 14, 2014}  \\
% The University of Notre Dame \\
% South Bend, IN
% \begin{itemize}\itemsep -2pt
%      \item A class presentation given as an author on a significant paper from the literature
% \end{itemize}

% {\bf Determining the Location of a Radioactive Source in \textsc{Majorana Demonstrator} \\
% $_{•}    $  \hfill August 2, 2011} \\
% {\sl Research Experience for Undergraduates Culminating Talks}  \\
% Duke University \\
% Durham, NC
% \begin{itemize}\itemsep -2pt
%      \item APS style presentation of summer research
% \end{itemize}



\section{Poster Presentations}
%  $_{•}    $\\
% \centerline{\bf Posters:}

% repeat of AAS 229th at GPSAC 2017 on September 13th at ND. http://gradphysics.nd.edu/conference/gpsac-2017/

\href{https://ui.adsabs.harvard.edu/#abs/2017AAS...22943402R/abstract}{{\sl Correlations Between Hubble Residuals and Local Stellar Populations}} \hfill {\bf January 7, 2017} \\
\textit{of Type Ia Supernovae}\\
AAS 229th Meeting, Grapevine, Texas

\href{https://ui.adsabs.harvard.edu/#abs/2016AAS...22723711R/abstract}{{\sl Correlating Type Ia Supernova Properties With Their Local}} \hfill {\bf January 6, 2016} \\
\textit{Environment Using HST Snapshots of Host Galaxies} \\
AAS 227th Meeting, Kissimmee, Florida

{\sl Prospects for Detecting a Cosmic Bulk Flow} \hfill {\bf January 6, 2015}\\
AAS 225th Meeting, Seattle, Washington  

{\sl Finding A Cosmic Bulk Flow}\hfill {\bf February 27, 2014}\\
GSU 6th Annual Research Symposium, Notre Dame 

{\sl Determining the Location of a Radioactive Source in \textsc{Majorana}} \hfill {\bf October 27, 2011}\\
\textit{\textsc{Demonstrator}} \\
APS, Division of Nuclear Physics, Michigan State University 
% $_{•}    $ \hfill {\bf October 27, 2011}

% {\sl Correlating Type Ia Supernova Properties with Their Local Environment Using HST Snapshots of Host Galaxies}. {\bf Rose, B.}, \& Garnavich, P. AAS 227th Meeting, Kissimmee, FL. January 6, 2016

% {\sl Prospects for Detecting a Cosmic Bulk Flow}. {\bf Rose, B.}, Garnavich, P., Mathews, G. J. AAS 225th Meeting, Seattle, WA. January 6, 2015

% {\sl Finding A Cosmic Bulk Flow}. {\bf Rose, B.}, Garnavich, P., Mathews, G. J. Graduate Student Union 6th Annual Research Symposium, University of Notre Dame, South Bend, IN. February 27, 2014

% {\sl Determining the Location of a Radioactive Source in \textsc{Majorana Demonstrator}} {\bf Rose, B.} APS, Division of Nuclear Physics. Michigan State University, East Lansing, MI. October 27, 2011

% American Physical Society, Department of Nuclear Physics\\ 
% Michigan State University \\
% East Lansing, MI 

% Old presentation method - like Talks, currently prefer posters to be like papers
% {\bf Correlating Type Ia Supernova Properties with Their Local Environment Using HST Snapshots of Host Galaxies  \hfill January 6, 2016} \\
% {\sl AAS 227th Meeting} \\
% Kissimmee, FL

% {\bf Prospects for Detecting a Cosmic Bulk Flow \hfill January 6, 2015} \\
% {\sl AAS 225th Meeting} \\
% Seattle, WA

% {\bf Finding A Cosmic Bulk Flow  \hfill February 27, 2014} \\
% {\sl Graduate Student Union 6th Annual Research Symposium} \\
% The University of Notre Dame \\
% South Bend, IN

% % \subsubsection{Poster Sessions:}
% % {\sl Finding A Cosmic Bulk Flow} \\
% % event Name \hfill February ??, 2014\\
% % University of Notre Dame \\
% % South Bend, IN

% \href{<http://meeting.aps.org/Meeting/DNP11/Session/EA.115>}{\bf Determining the Location of a Radioactive Source in \textsc{Majorana Demonstrator}} \\
% {\bf $_{•}    $  \hfill October 27, 2011 }\\
% American Physical Society, Department of Nuclear Physics\\ 
% Michigan State University \\
% East Lansing, MI 




%todo(should this link to newest on top, or most cited on top?)
\section{\href{https://ui.adsabs.harvard.edu/#search/q=orcid\%3A\%220000-0002-1873-8973\%22&sort=date\%20desc\%2C\%20bibcode\%20desc}{Publications}}

% \begin{etaremune}%[align=left]

\hangindent=15pt [4] {\sl Detectability of Cosmic Dark Flow in the Type Ia Supernova Redshift-Distance Relation}\\
\href{https://ui.adsabs.harvard.edu/#abs/2016ApJ...827...60M/abstract}{Mathews, G.J., {\bf Rose, B. M.}, Garnavich, P., et al. 2016 ApJ 827 60}

\hangindent=15pt [3] {\sl The New Eclipsing CV MASTER OTJ192328.22+612413.5: A Possible SW Sextantis Star}\\
\href{https://ui.adsabs.harvard.edu/#abs/2016AJ....152...27K/abstract}{Kennedy, M.~R., Callanan, P., Garnavich, P.~M., {\bf et al.} 2016, AJ, 152, 27}

\hangindent=15pt [2] {\sl Constraints on the Birth of the Universe and Origin of Cosmic Dark Flow}\\
\href{https://ui.adsabs.harvard.edu/#abs/2015IJMPA..3045022M/abstract}{Mathews, G.~J., Gangopadhyay, M.~R., Garnavich, P., {\bf Rose, B. M.}, et al. 2015 Int. J. Mod. Phys. A, 30, 1545022}

\hangindent=15pt [1] {\sl High-Amplitude, Rapid Photometric Variation of the New Polar MASTER OT J1321}\\
\href{https://ui.adsabs.harvard.edu/#abs/2015IBVS.6129....1L/abstract}{Littlefield, C., Garnavich, P., Magno, K., {\bf et al.}\ 2015, Information Bulletin on Variable Stars, 6129, 1}

% \end{etaremune}
% \newpage
% \addtocounter{section}{1}
% \phantomsection
% \addcontentsline{toc}{section}{Title of the section (my papers?)}
% \bibliographystyle{acm}
% \bibliography{mendeley}
% \nocite{*}
% % \nocite{Mathews2014}



% change this to outreach(/leadership maybe) & research interests.

% \section{Lab Work\\ \& \\Projects}
% {\sl Searching for a Cosmic Bulk Flow} \hfill ?? 
% \begin{itemize}\itemsep -2pt
%      \item Searching for a Dark Flow signal using supernovae (SNe)
%      \item Working with multiple SNe catalogs
%      \item Developing my own analysis code
% \end{itemize}

% {\sl Physics Outreach} \hfill Spring 2012 
% \begin{itemize}\itemsep -2pt
% 	\item Worked with middle school students on high altitude weather balloon experiments
% 	\item Personally worked on radiation and ozone detector systems, as well as temperature sensors
% \end{itemize}

% {\sl Determining the Location of a Radioactive Source in \textsc{Majorana Demonstrator}}
%   \\ $_{•}    $                   \hfill Summer 2011
% \begin{itemize}\itemsep -2pt
% 	\item Presented at poster session of APS DNP 2011 
% 	\item Worked at TUNL analyzing Monte Carlo generated data to determine how well  \textsc{Majorana} can resolve the location of a radioactive hot spot
% \end{itemize}

% {\sl Cosmic Rays} \hfill Fall 2010 
% \begin{itemize}\itemsep -2pt
% 	\item Detected Cosmic Rays with respect to altitude using a three Geiger counter telescope
% 	\item Used a high altitude weather balloon
% \end{itemize}

% {\sl Electronic Robot} \hfill Spring 2010 
% \begin{itemize}\itemsep -2pt
% 	\item Electronics class final project
% 	\item Designed and built a robot that wrote and erased on a whiteboard
% \end{itemize}

% {\sl Differential Equations} \hfill Fall 2009
% \begin{itemize}\itemsep -2pt
% 	\item Differential Equations class final project
% 	\item Mapping the movement of a building during an earthquake by using numerical methods in MATLAB 
% \end{itemize}


\end{resume}
\end{document}






%%%%%%%%%%%%%%%%%%%%%%%%%%%%%%%%%%%%%%%%%%%%%%%%%%%%%%%%%%%%%%%%%%%%%%%%%%%%%%%%%%%%%%%%%%%%
%%%%%%%%%%%%%%%%%%%%%%%%%%%%%%%%%%%%%%%%%%%%%%%%%%%%%%%%%%%%%%%%%%%%%%%%%%%%%%%%%%%%%%%%%%%%
%%%%%%%%%%%%%%%%%%%%%%%%%%%%%%%%%%%%%%%%%%%%%%%%%%%%%%%%%%%%%%%%%%%%%%%%%%%%%%%%%%%%%%%%%%%%
\begin{comment}
\entry{Research Assistant \hfill January 2014 - Present\\}
{\bf Research Assistant \hfill January 2014 - Present}
The University of Notre Dame 
     
{\sl Teacher Assistant} \hfill August 2012 - December 2013 \\
The University of Notre Dame 
\begin{itemize}\itemsep -2pt
\item More detail under Teaching Experience.
\end{itemize}

{\sl Supplemental Instruction (SI) Leader for Physics} \hfill September 2011 - May 2012 \\
 Whitworth University
\begin{itemize}  \itemsep -2pt %reduce space between items
     \item Led a    group study session of introductory physics material
     \item Implemented active learning techniques
\end{itemize}

{\sl Research Experience for Undergraduates} \hfill Summer 2011 \\
Triangle Universities Nuclear Laboratory (TUNL) \\ 
Durham, NC
\begin{itemize}  \itemsep -2pt %reduce space between items
     \item Worked on \textit{Determining the Location of a Radioactive Source in \textsc{Majorana Demonstrator}} described in ``Lab Work \& Projects" 
\end{itemize}

{\sl Laboratory Teaching Assistant} \hfill February 2011 - May 2011 \\
 Whitworth University
\begin{itemize}  \itemsep -2pt %reduce space between items 
     \item Assisted in and graded an introductory physics lab session
\end{itemize}

{\sl Personal Tutor} \hfill September 2010 - May 2011 \\
 High School Student, Algebra II and Geometry
\begin{itemize}  \itemsep -2pt %reduce space between items
     \item One-on-one sessions, reinforcing content and working on homework
\end{itemize}

{\sl Daycare Teacher} \hfill January 2009 - May 2011 \\
 Whitworth Community Presbyterian Church
\begin{itemize}  \itemsep -2pt %reduce space between items
     \item Responsible for up to 10 preschoolers, worked as a part of a team, planned kids activities, part of closing team, created a safe environment, encouraged creativity, and reinforced developmental skills
\end{itemize}

{\em University of Notre Dame:}
\begin{itemize} \itemsep -2pt %reduce space between items
     \item Laboratory Teaching Assistant and Grader
     \begin{itemize} \itemsep -2pt %reduce space between items
          \item Physics 1 for Life Science \hfill Fall 2013, Summer 2013
          \item Phycics 2 for Engineers \hfill Spring 2013
          \item Physics 2 for Life Science \hfill Fall 2012
     \end{itemize}
     \item Exam Grader \hfill Fall 2012 - Fall 2013
\end{itemize}
{\em Whitworth University:}
\begin{itemize} \itemsep -2pt %reduce space between items
     \item SI Instructor
     \item Laboratory Teaching Assistant and Grader
\end{itemize}

personal tutor:One-on-one sessions, reinforcing content and working on homework
\end{comment}