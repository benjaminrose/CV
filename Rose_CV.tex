% LaTeX resume using res.cls
\documentclass[margin]{res}
\renewcommand*\sfdefault{uop}
% \setlength{\textwidth}{5.2in}%{5.2in} % set width of text portion


%%%%%%%%
%Imports
%%%%%%%%

\usepackage{verbatim} %lets me comment out blocks of text with \begin{comment} and \end{comment}
\usepackage{cleveref}
\usepackage{amssymb,graphicx}
\usepackage{etaremune}  %reverse counting lists, http://texblog.org/2011/11/29/reverse-enumerate-or-etaremune/
\usepackage[ampersand]{easylist}
% \usepackage{blindtext}
\usepackage{enumitem}
\usepackage{color}   %to grey out the numbers for my publication list
\definecolor{light-gray}{gray}{0.55}
\newcommand{\todo}[1]{\textbf{\textcolor{red}{[[#1]]}}\xspace}


%%%%%%%%
%Links
%%%%%%%%

% \usepackage[unicode]{hyperref}% add hypertext capabilities
% \urlstyle{same}

%https://tex.stackexchange.com/questions/53300/how-to-disable-links-completely-using-hyperref-package
\usepackage{nohyperref}   % turns off links, 
\usepackage{url}


%%%%%%%%
%Fonts
%%%%%%%%

% \usepackage{charter}
\usepackage{fontspec}   %need to use LuaTEX or XeLaTeX
% \defaultfontfeatures{Mapping=tex-text,Scale=MatchLowercase}
% \setmainfont{Charter}
% \setmainfont{GaramondNo8-Regular.ttf}[
%     BoldFont = GaramondNo8-Bold.ttf ,
%     ItalicFont = GaramondNo8-Italic.ttf ,
%     BoldItalicFont = GaramondNo8-Bold-Italic.ttf]
% \setboldfont{GaramondNo8-Bold}
% \setmonofont{Lucida Sans Typewriter}


%%%%%%%%
%Footer
%%%%%%%%

%This lets me add headers/footers
\usepackage{fancyhdr} % need to use xelatex or lualalex typesetting engine
\pagestyle{fancy}
\fancyhf{}
% \thispagestyle{\rhead{\includegraphics[scale= 0.18]{BenRose_ND_Contact.png}}}

\addtolength{\oddsidemargin}{-.25in}
\addtolength{\evensidemargin}{-.25in}
\addtolength{\textwidth}{0.5in}
\addtolength{\resumewidth}{0.5in}
% \setlength{\parindent}{15pt}

\renewcommand{\headrulewidth}{0.0pt}
%\lhead{}
%\chead{}
%\rhead{Benjamin Rose CV \thepage} %\thepage} this adds a page number
%\rhead{\includegraphics[scale= 0.15]{BenRose_ND_Contact.png}}
\lfoot{\small{last updated: \today}}
\cfoot{}
\rfoot{\small{Benjamin M. Rose, page \thepage}}
%\rhead{\includegraphics[scale= 0.18]{BenRose_ND_Contact.png}}

\thispagestyle{fancy}
\thispagestyle{empty}
%\rhead{\includegraphics[scale= 0.18]{BenRose_ND_Contact.png}}

\makeatletter
\newcommand\entry{\@startsection{subsubsection}{3}{\z@}%
                                     {-3.25ex\@plus -1ex \@minus -.2ex}%
                                     {-1.5ex \@plus -.2ex}% Formerly 1.5ex \@plus .2ex
                                     {\normalfont\normalsize\bfseries}}
\makeatother



\begin{document}
% Center the name over the entire width of resume:
\moveleft.5\hoffset\centerline{\huge \bf Benjamin M. Rose}
% \moveleft.5\hoffset\centerline{\large \bf Physics Ph.D. Candidate at Notre Dame}
\vspace{2pt}
\moveleft.5\hoffset\centerline{ \href{http://orcid.org/0000-0002-1873-8973}{ORCID: 0000-0002-1873-8973}}
% Draw a horizontal line the whole width of resume:
\setlength{}{}
\moveleft\hoffset\vbox{\hrule width\resumewidth height 1pt}\smallskip
% address begins here
% Again, the address lines must be centered over entire width of resume:
\moveleft.5\hoffset\centerline{Department of Physics, Duke University \hfill (919) 660-2500}
\moveleft.5\hoffset\centerline{Campus Box 90305 \hfill \href{mailto:brose@stsci.edu}{\url{benjamin.rose@duke.edu}}}
\moveleft.5\hoffset\centerline{Durham, North Carolina 27708 \hfill }%\href{http://www.nd.edu/~brose3}{www.nd.edu/\~{}brose3}}
%\moveleft.5\hoffset\centerline{574.387.3453 -- \url{brose3@nd.edu}}
%\moveleft.5\hoffset\centerline{\url{www.nd.edu/~brose3}}
%\moveleft.5\hoffset\centerline{brose3@nd.edu} %try to make clickable using /url??
%\moveleft.5\hoffset\centerline{brose12@my.whitworth.edu}




\begin{resume}

\section{Employment} 

{\bf Research Scientist} \hfill {\bf 2020--present}\\
{\it Department of Physics, \href{https://phy.duke.edu/research/research-areas/astrophysics}{Cosmology Research Group}}\\
\href{https://phy.duke.edu}{Duke University}, Durham, North Carolina\\ 
Supervisor: \href{https://phy.duke.edu/people/daniel-m-scolnic}{Assistant Professor Dan Scolnic}

{\bf Postdoctoral Fellow} \hfill {\bf 2018--2020}\\
% {\it Roman Space Telescope Supernova Science} \\
\href{http://www.stsci.edu}{Space Telescope Science Institute}, Baltimore, Maryland\\ 
Supervisors: Drs. Susana Deustua and Andrew Fruchter

% \href{http://physics.nd.edu}{\bf University of Notre Dame} {\bf \hfill August 2012 - Present}\\
% {\sl Research Assistant} \hfill 2 semester \\
% {\sl \href{labTA}{Teaching Assistant} \cref*{SI} } \hfill 3 semesters
% %todo(make ref to different section in this text)

% \href{http://www.whitworth.edu/physics/}{\bf Whitworth University} {\bf \hfill February 2011 - May 2012}\\
% {\sl Supplemental Instruction Leader \cref{SI} for Physics } \hfill 2 semesters \\
% {\sl Laboratory Teaching Assistant } \hfill 1 semester
% % \begin{itemize}\itemsep -2pt
% % \item More details under Teaching Experience
% % \end{itemize} 
 
% % {\bf Personal Tutor \hfill September 2011 - May 2012}\\
% % {\sl High School Geometry and Algebra } 

% \href{http://www.tunl.duke.edu}{\bf Triangle Universities Nuclear Laboratory} {\bf \hfill Summer 2011}\\
% {\sl Research Experience for Undergraduates }
% \begin{itemize}\itemsep -2pt
% \item Durham, NC
% % \item Worked on \textit{Determining the Location of a Radioactive Source in \textsc{Majorana Demonstrator}} described in Lab Work \& Projects 
% \end{itemize} 




\section{Education}

{\bf Doctor of Philosophy in Physics} \hfill {\bf 2018}\\
% \textit{Type Ia Supernovae: Testing the Cosmological Principle and Improving \\Distance Measurements With Implications for the Hubble Constant}\\
{\bf Master of Science in Physics} \hfill  {\bf 2016}\\
% {\it title} \\
\href{http://physics.nd.edu}{University of Notre Dame}, Notre Dame, Indiana \\ 
Advisor: \href{http://www.nd.edu/~pgarnavi}{Professor Peter Garnavich}


{\bf Bachelor of Science in Physics,} \textit{cum laude} \hfill \textbf{2012}\\
% cum laude\\%, GPA: 3.6/4.0 \\
\href{http://www.whitworth.edu/physics/}{Whitworth University}, Spokane, Washington\\
Minor: Mathematics
 


% \section{\href{http://www.nd.edu/~brose3/current-research.html}{Research \\Interests}}
% % \begin{tabular}{l l}
% % Type Ia Supernovae & Observational Cosmology  \\
% % Type Ia Supernovae Host Environment & Nuclear Astrophysics \\
% % Public and Community Based Science
% % \end{tabular}

% \begin{itemize}\itemsep -2pt
% \item Type Ia supernovae%, specifically host environments
% \item Observational cosmology 
% \item Public and open based science
% \end{itemize} 



%this should just be on the web, it seems odd in a CV.
% \section{Research \\Experience}
% {\bf Searching for a Cosmic Bulk Flow} 
% \begin{itemize}\itemsep -2pt
%      \item Minimized around a cosine distribution to look for a directional dependence to the residuals of the Hubble Diagram
%      \item Developed in Python with \href{http://www.astropy.org}{astropy}, \href{http://roban.github.io/CosmoloPy/}{CosmoloPy}, and \href{https://code.google.com/p/pyminuit/}{PyMinuit}
% \end{itemize}

% {\bf Determining the Location of a Radioactive Source in \textsc{Majorana Demonstrator}}
% \begin{itemize}\itemsep -2pt
%    \item Worked at \href{http://www.tunl.duke.edu}{TUNL} analyzing simulated data with \href{http://root.cern.ch/drupal/content/pyroot}{pyROOT} to determine how well the \textsc{Majorana} detector array can resolve the location of a radioactive hot spot.
% \end{itemize}



\section{Awards \& \\Grants} 

\textbf{Astrophysics Research and Analysis Proposal}, NASA\hfill {\bf 2022--2026}
\begin{itemize}  \itemsep -2pt %reduce space between items
     \item[] Collaborator, \textit{CANDLE: Calibration using an Artificial star with\\NIST-traceable Distribution of Luminous Energy}
     \item[] \$1.5 Million over 4 years
    \end{itemize} \vspace{-12pt}
\textbf{Astronomy and Astrophysics Research Grants}, NSF \hfill \textbf{2022--2024}
\begin{itemize}  \itemsep -2pt %reduce space between items
%https://www.nsf.gov/awardsearch/showAward?AWD_ID=2205635&HistoricalAwards=false
     \item[] Co-Investigator, \textit{Are Hubble Residuals a Product of Poor Mass\\Estimates? Improving Supernova Ia Host Galaxy Characterizations} 
     \item[] \$355,000
    \end{itemize} \vspace{-12pt}
\textbf{HST-GO Grant, Cycle 30}, Space Telescope Science Institute \hfill \textbf{2022--2023}
\begin{itemize}  \itemsep -2pt %reduce space between items
     \item[] Program Admin PI, \textit{Local Environment of Low-redshift \\Type Ia Supernova Siblings} 
     \item[] \$56,000
    \end{itemize} \vspace{-12pt}
{\bf Lennox Graduate Fellowship}, Notre Dame \hfill {\bf 2017}
    \begin{itemize}  \itemsep -2pt %reduce space between items
     \item[] Recognizes achievements and promise as a graduate student in physics
    \end{itemize} \vspace{-12pt}
{\bf Conference Presentation Grant}, Notre Dame, Graduate Student Union  \hfill {\bf 2015 \& 2016}\\
{\bf Notebaert Professional Development Award}, Notre Dame \hfill {\bf 2015 \& 2016}\\
{\bf Poster Grant}, GSU 6th Annual Research Symposium \hfill {\bf 2014}
% {\bf Presidential Scholarship}, Whitworth University \hfill {\bf2008--2012}\\
% {\bf Delbert E. Friesen Memorial Scholarship}, Whitworth University \hfill {\bf2011--2012}\\
% {\bf Talent Scholarship in Physics}, Whitworth University \hfill {\bf2008--2011}
% \\
% {\bf Talent Scholarship in Music}, Whitworth University \hfill {\bf2009 - 2012}\\
% {\bf Laureate Society}, Whitworth University \hfill {\bf4 semesters}
% %\\For a semester GPA of 3.75 or greater
% \begin{itemize}  \itemsep -2pt %reduce space between items
%      \item[] \textit{\small{For a semester GPA of 3.75 or greater}}
% \end{itemize} %\vspace{-12pt}






\section{Observational Programs} % and grants/proposals?

% {\bf Co-I ADAP  {\bf 2021}
%     \begin{itemize} \itemsep -2pt %reduce space between items
%      \item[] \textit{idk}
%      \end{itemize} \vspace{-12pt}

%SNAP 16682
{\bf Principal Investigator, Gemini 2022B} \hfill \textbf{2022} 
\begin{itemize} \itemsep -2pt %reduce space between items
     \item[] \textit{Constraining a Redshift Dependent Type Ia Supernova Mass Step with \\Improved Stellar Mass Measurements}
     \item[]\href{http://www.gemini.edu/observing/schedules-and-queue/2022b-gs-queue-band-1-3}{Gemini South (GS-2022B-Q-306), awarded 19.8 hours}
     \item[]\href{http://www.gemini.edu/observing/schedules-and-queue/2022b-gn-queue-band-4}{Gemini North (GN-2022B-Q-404), awarded 10.8 hours}
    %  \item[]\href{http://www.sdss.org/dr14/manga/manga-target-selection/ancillary-targets/}{SOAR Program, awarded 1 night}
     \end{itemize} \vspace{-12pt}

\newpage
{\bf Co-Investigator, HST Cycle 30 \& 31} \hfill {\bf 2022-2023}
\begin{itemize} \itemsep -2pt %reduce space between items
     \item[] \textit{Reducing Type Ia Supernova Distance Biases by Separating Reddening\\and Intrinsic Color}
    %  Proposal ID 16682,
     \item[] \href{https://archive.stsci.edu/proposal_search.php?id=17128&mission=hst}{Awarded 135 orbits}
     \end{itemize} \vspace{-12pt}
     
{\bf Program Admin PI, HST Cycle 30 Snapshot} \hfill {\bf 2022}
% {\bf Co-Investigator, HST Cycle 30 Snapshot} \hfill {\bf 2022}
\begin{itemize} \itemsep -2pt %reduce space between items
     \item[] \textit{Local Environments of Low-redshift Type Ia Supernova Siblings}
    %  Proposal ID 16682,
     \item[] \href{https://archive.stsci.edu/proposal_search.php?id=17194&mission=hst}{Awarded 32 targets}
     \end{itemize} \vspace{-12pt}


{\bf Co-Investigator, HST Cycle 29 Snapshot} \hfill {\bf 2021}
\begin{itemize} \itemsep -2pt %reduce space between items
     \item[] \textit{UV Spectroscopy of Astronomical Transients through Rolling Snapshots}
    %  Proposal ID 16682,
     \item[] Awarded 100 targets
     \end{itemize} \vspace{-12pt}
% \\ \textit{UV Spectroscopy of Astronomical Transients through Rolling Snapshots}\\
% Awarded 100 orbits.
% \vspace{-12pt}

{\bf Co-Investigator, Spitzer DDT Program} \hfill {\bf 2019}
    \begin{itemize} \itemsep -2pt %reduce space between items
     \item[] \textit{IRAC Photometry for the Cosmic Flux Standards - A Network of Faint \\Absolute Calibrators}
     \end{itemize} \vspace{-12pt}


{\bf Principal Investigator, \href{http://www.sdss.org/dr13/manga/}{SDSS-IV MaNGA Ancillary Program}} \hfill \textbf{2017} 
\begin{itemize} \itemsep -2pt %reduce space between items
     \item[] \href{https://trac.sdss.org/wiki/MANGA/Survey/AncillaryPrograms2017/Rose_SNIa_Environments_and_HR}{\textit{Exploring a Possible Correlation Between Hubble Residuals and SN Ia\\Local Environments}}
     \item[]\href{http://www.sdss.org/dr14/manga/manga-target-selection/ancillary-targets/}{Awarded 40 ancillary targets}
     \end{itemize} \vspace{-12pt}
% \\
% \href{https://trac.sdss.org/wiki/MANGA/Survey/AncillaryPrograms2017/Rose_SNIa_Environments_and_HR}{\textit{Exploring a Possible Correlation Between Hubble Residuals and SN Ia Local Environments}}\\
% \href{http://www.sdss.org/dr14/manga/manga-target-selection/ancillary-targets/}{Awarded 40 ancillary targets}
{\href{http://vaticanobservatory.org/VATT/}{\bf  Vatican Advanced Technology Telescope (VATT)}} \hfill \textbf{2014}
\begin{itemize} \itemsep -2pt %reduce space between items
     \item[] \href{http://mgio.arizona.edu}{Mount Graham International Observatory}, Safford, Arizona
     \item[]4 nights
     \end{itemize} 
% \begin{itemize}\itemsep -2pt
%   \item[] June 27, 2014 - July 1, 2014
% \end{itemize}




\section{Activities \& Service} %Service and outreach?? - Referee is often service.

% \hspace{-1.5em}\begin{tabular*}{\resumewidth}{p{4.17in} r}
% \begin{description}[align=left] % use \itme[-]
% \ListProperties(Style2*=,Numbers=a,Numbers1=R,FinalMark={)})

\textbf{\href{https://roman.gsfc.nasa.gov/science/2022_cosmology_seminar.html}{Cosmology with the Nancy Grace Roman Space Telescope}} \hfill \textbf{January 2022}
\begin{itemize}\itemsep -2pt
    \item[] Chair of the organizing committee
    \item[] A virtual seminar from a cancelled 237th AAS session
    % \item[] We organized a conference with over 60 attendees
\end{itemize} \vspace{-12pt}
\textbf{Roman RCS Under-performance Mitigation Task Force}, member \hfill \textbf{September 2020}\\
\textbf{\href{https://www.stsci.edu/contents/events/stsci/2020/march/accurate-flux-calibration-for-21st-century-astrophysics?timeframe=}{Accurate Flux Calibration}}, organizing committee, \textit{cancelled} \hfill \textbf{March 2020}\\
\textbf{WFIRST Science Jamboree}, organizing committee \hfill \textbf{July 2019}\\
\textbf{WFIRST Simulated Data Hack Day}, organizing committee \hfill \textbf{March 2019}\\
\textbf{\href{gradphysics.nd.edu}{Graduate Physics Society}}
\begin{itemize}\itemsep -2pt
    \item[] \href{http://gradphysics.nd.edu/about-us/executive-board/}{Executive board member} \hfill {\bf 2015--2017}
    \item[] \href{http://gradphysics.nd.edu/about-us/committee-chairs/}{Public relations chair} \hfill {\bf 2017--2018}
    \item[] \href{http://gradphysics.nd.edu/conference/gpsac-2016/}{Annual conference co-chair}
    %, with over 60 attendees 
    \hfill {\bf 2016}
    \item[] Member \hfill {\bf 2012--2018}
    % \item[] \hspace{1.5em} We organized a conference with over 60 attendees
\end{itemize} \vspace{-12pt}
% {\bf \href{gradphysics.nd.edu}{Graduate Physics Society (GPS) PR Chair}} \hfill {\bf 2017 - present}\\
% {\bf \href{gradphysics.nd.edu}{Graduate Physics Society (GPS) Executive Board Member}} \hfill {\bf 2015 - 2017}
% \\
    % \begin{itemize}\itemsep -2pt
    % \item[] \textit{\small{Executive Board Member promoting the Socieity's goal of ``fostering a \\community built on intellectual, professional, and social interactions.''}}
    % \end{itemize} \vspace{-12pt}
% {\bf \href{http://gradphysics.nd.edu/conference/gpsac-2016/}{Graduate Physics Society Annual Conference Co-chair}} \hfill {\bf 2016 }
% % \\Member of the Oranizing Committee\\
%     \begin{itemize}\itemsep -2pt
%     \item[] \textit{\small{We organized a conference with over 60 attendees.}} % A total of 16 talks and posters
%     \end{itemize} \vspace{-12pt}
%\href{http://gradphysics.nd.edu/2016/11/30/gpsac-2016-agenda/}{The conference had over 10 posters \\and 6 talks.}
    % \begin{itemize}\itemsep -2pt
    % \item[] Organizing Committee Member 
    % \end{itemize} \vspace{-10pt}   
{\bf Graduate Student Union Representative} \hfill {\bf 2013--2014}
%\\ \textit{\small{I was the Physics Department representative and worked on issues of parking, \\health insurance, building remodels, and more.}}\\
    % \begin{itemize}\itemsep -2pt
    % \item[] Advocated for issues including health insurance and office spaces
    % \end{itemize} %\vspace{-12pt}
% {\bf Notre Dame Summer Band} \hfill {\bf 2013 \&  2014 }\\
% {\bf Whitworth University Wind Symphony}  \hfill {\bf 2009 - 2012 }\\
% {\bf\href{http://www.whitworthnearspace.org/wiki/Main_Page}{Whitworth Near Space}} \hfill  \textbf{Spring 2012} 
% % \\I worked with middle school students on high altitude balloon \\experiments.
% % We developed and built radiation, ozone, and temperature \\detector systems. \\
%     \begin{itemize}\itemsep -2pt
%     % \item[] Worked with middle school students on high altitude balloon experiments
%     % \item[] I worked with middle school students on two high altitude balloon \\experiments. We developed and built radiation, ozone, and \\temperature detector systems.
%     \item[] Assisted middle school students on two high altitude balloon \\experiments by assisting in building radiation, ozone and temperature \\sensors
%     % \item[] Developed \& built radiation, ozone, and temperature detector systems
%     \end{itemize}\vspace{-12pt}
% {\bf Club Treasurer}, Whitworth University \hfill {\bf2009--2011}
% {\bf Massah} \hfill Summer 2010
% \begin{itemize}  \itemsep -2pt %reduce space between items
%      \item A 10 week program of developing and studying group dynamics and cross-cultural activities in Israel and India
% \end{itemize}
% {\bf Small Group Leader}, Whitworth University \hfill 2009 - 2010
% \begin{itemize}  \itemsep -2pt %reduce space between items
%      \item Co-leading of a student Bible study
% \end{itemize}

% \end{tabular*}
% \end{description}




\section{Mentoring} % combine with teaching?

\textbf{Kevin Wang}, Duke University \hfill \textbf{2021--present}
\begin{itemize}\itemsep -2pt
    \item[] Ph.D. student working on measuring the effect of weak gravitational\\lensing on supernovae cosmology, with a specific focus the Roman\\Space Telescope.
    \end{itemize}
    \vspace{-12pt}
\textbf{Kayla Perkinson}, Space Telescope Science Institute \hfill \textbf{2019}
    \begin{itemize}\itemsep -2pt
    \item[] High school intern working on a project to understand the capabilities\\of the Romans Space Telescope prism.
    \end{itemize}

    


\section{\href{https://www3.nd.edu/~brose3/\#classes}{Teaching Experience}}\label{teaching}
\textbf{\href{https://www3.nd.edu/~brose3/2017reu-cmp}{Introduction to Scientific Computing with Python}} \hfill{} \textbf{Summer 2017 \& 2018}
\begin{itemize}\itemsep -2pt
    \item[] Lead instructor and supervisor of teaching assistants for the Notre \\Dame REU program
    \end{itemize} \vspace{-12pt}
\textbf{\href{https://www3.nd.edu/~brose3/2017reu-gre}{Physics GRE Preparation Course}} \hfill{} \textbf{Summer 2017 \& 2018}
\begin{itemize}\itemsep -2pt
    \item[] Instructor for an exam review course for the Notre Dame\\REU program
    \end{itemize} \vspace{-12pt}
\textbf{Private Tutor} \hfill \textbf{2016--2019}
\begin{itemize}\itemsep -2pt
    \item[] Tutored ten students in introductory physics (mechanics and E\&M),\\modern physics, thermodynamics, math methods and statistics at both\\the high school and undergraduate level.
    \end{itemize} \vspace{-12pt}
% \textbf{Individual Tutoring} \hfill{} \textbf{Fall 2015 \& 2016}
% % \\Sessions included topics in math methods, thermodynamics, and E\&M.\\
%     \begin{itemize}\itemsep -2pt
%     \item[] \textit{\small{Sessions included topics in math methods, thermodynamics, and E\&M.}}
%     \end{itemize} \vspace{-12pt}
\textbf{Introduction to Scientific Computing with Python} \hfill{} \textbf{Spring 2016}
% \\I taught weekly basic programing help session and graded programing\\ assignments.\\
    \begin{itemize}\itemsep -2pt
    \item[] Organized help session and graded assignments
    \end{itemize} \vspace{-12pt}
% \textbf{various exam grading}
% \textbf{Homework grading} \hfill{} \textbf{A lot}
%     \begin{itemize}\itemsep -2pt
%     \item[] Intro. to Astronomy, Calculus based
%     \item[] Intro. to Astronomy, no math (maybe Intro. to Cosmology)
%     \end{itemize} \vspace{-10pt}
\textbf{Engineering Introductory Physics Labs} \hfill{} \textbf{Spring \& Fall 2013}\\
\textbf{Pre-Medical Introductory Physics Labs} \hfill{} \textbf{Fall 2012 \& Summer 2013}\\
% $_{•}    $ \hfill \textbf{Summer 2013}
    % \begin{itemize}\itemsep -2pt
    % \item[] Three semesters, multiple sections
    % \end{itemize} \vspace{-10pt}
{\bf Supplemental Instruction Leader} \hfill{} \textbf{Fall 2011 \& Spring 2012}
    \begin{itemize}\itemsep -2pt
    \item[] Led peer-to-peer study sessions focused on active learning techniques
    \end{itemize}


% {\bf Personal Tutor:}
% Worked individually with a student on material and study skills for High School Geometry and Algebra.
\begin{comment}
year |   Fall        | Spring        | Summer
1st  | Labs-premed   | labs-eng      | labs-premed
2nd  | Labs-eng      | RA            | RA
3rd  | RA/IntroAstro | Peter's intro | RA
4th  | IntroAstro    | Python        | RA
5th  | Physics C HW  | RA            | RA & REU classes
6th  | Fellowship (GRE Class) | Python | NA!?!
\end{comment}






\section{Reviewer}

The Astrophysical Journal, Letters \\ %~$\diamond$~
\href{https://github.com/openjournals/joss-reviews/issues?utf8=✓&q=is\%3Aissue+benjaminrose+label\%3Areview}{Journal of Open Source Software} \\ 
Monthly Notices of the Royal Astronomical Society, Letters
\\ %~$\diamond$~ 
NASA Research Opportunities in Space and Earth Sciences (ROSES) \\
National Research, Development and Innovation Office, Hungary





\section{Science Collaborations}
{\bf Nancy Grace Roman Space Telescope} (formerly WFIRST)
\begin{itemize}\itemsep -2pt
 \item[] Foley Supernova Science Investigation Team \hfill {\bf 2020--2022}
 \item[] Perlmutter Supernova Science Investigation Team \hfill {\bf 2018--2022}
\end{itemize} \vspace{-12pt}
{\bf \href{http://www.lsst-desc.org}{LSST Dark Energy Science Collaboration,}} Member \hfill {\bf 2018--present}
\\
% \begin{itemize}\itemsep -2pt
%  \item[] Time Domain Analysis Working Group Co-Convener \hfill {\bf 2021--present}
% \end{itemize} \vspace{-12pt}
{\bf Dark Energy Survey}, Member \hfill {\bf 2020--present}\\
{\bf Sloan Digital Sky Survey (SDSS) V,} Member \hfill {\bf 2019--2020}\\
{\bf Sloan Digital Sky Survey IV,} Member \hfill {\bf 2014--2018}
% {\bf SDSS-III} Member \hfill {\bf 2012--2014} %technically





\section{\href{https://github.com/benjaminrose}{Open Source}}

% \hspace{-1em}
%% There are three levels here. Contributed to source, contributed to documentaiton, and contributed to the community.
% \begin{tabular}{l p{4.9in}}
% \href{https://www.scipy.org}{\texttt{scipy}} & \href{https://github.com/scipy/scipy/pull/8011}{documentation update of limitation in \texttt{integrate.quad}} accepted\\
% \href{https://www.scipy.org}{\texttt{corner.py}} & \href{https://github.com/scipy/scipy/pull/8011}{[WIP] Update to title API}\\
% \href{http://sep.readthedocs.io/en/v1.0.x/}{\texttt{sep}} & \href{https://github.com/kbarbary/sep/commit/612033788bcce44f110a87e1b54bb70eea9960c2}{documentation update accepted} \\
% \href{http://dan.iel.fm/emcee/current/}{\texttt{emcee}} & \href{https://github.com/dfm/emcee/pull/212}{documentation update accepted} \\
% \href{https://seaborn.pydata.org}{\texttt{seaborn}} & \href{https://github.com/mwaskom/seaborn/pull/1407}{documentation update accepted}, \href{https://github.com/mwaskom/seaborn/pull/1531}{twice.}\\
% \href{http://www.astropy.org}{\texttt{astropy}} & \href{https://github.com/astropy/astropy/issues/4976}{reported an issue in World Coordinate System utility} 
% \\
% \texttt{snemo} & Citation update accepted
% % \\
% % \href{https://github.com/astrofrog/acknowledgment-generator}{Acknowledgment Generator} & \href{https://github.com/astrofrog/acknowledgment-generator/pull/63}{Added SDSS IV}
% \end{tabular}


\begin{tabular}{l p{4.3in}}
\hspace{-0.6em}{\bf Co-maintainer:} & SNCosmo\\
\hspace{-0.6em}{\bf Source Code:} & 
corner.py, kde\_corner, SNANA, \href{https://github.com/conda-forge/extinction-feedstock}{extinction}, \href{https://github.com/rubind/host_unity}{UNITY}\\

\hspace{-0.6em}{\bf Documentation:} & 
emcee,
\href{https://github.com/scipy/scipy/pull/8011}{scipy},
\href{https://github.com/mwaskom/seaborn/pulls?q=is\%3Apr+author\%3Abenjaminrose}{seaborn},
\href{https://github.com/kbarbary/sep/commit/612033788bcce44f110a87e1b54bb70eea9960c2}{sep}\\

% \hspace{-1em}{\bf Community Member:} & 
\end{tabular}







% \newpage
\section{\href{https://github.com/benjaminrose}{Technical \\Skills}}

%{\bf Computer Languages \& Programs:}
\hspace{1pt}
\begin{tabular}{l p{4.5in}}

%alternative headers: Extensive Experience, Some experience, familiar with
\hspace{-1em}{\bf Daily Use:} & 
\texttt{bash},
\texttt{click},
\href{https://www.python.org}{Python},
\texttt{numpy},
\texttt{scipy},
macOS,
\LaTeX,
\href{https://git-scm.com}{\texttt{git}},
\href{https://github.com/benjaminrose}{GitHub},
\href{http://www.astropy.org}{Astropy},
\href{http://daringfireball.net/projects/markdown/}{Markdown},
matplotlib,
Overleaf,
\href{http://pandas.pydata.org}{\texttt{pandas}},
\texttt{zsh}
\\

\hspace{-1em}{\bf Proficient:} & 
\texttt{codcov},
Confluence,
\href{http://dan.iel.fm/emcee/current/}{\texttt{emcee}},
GitHub Actions,
\href{https://jekyllrb.com}{Jekyll},
\href{http://jupyter.org}{Jupyter Notebook},
\\&
\href{https://jupyterlab.readthedocs.io/en/stable/}{JupyterLab},
\texttt{pytest},
\texttt{pymc3},
reStructuredText,
\href{http://mc-stan.org}{\texttt{stan}},
\texttt{scikit-learn},
\\&
TravisCI\vspace{0.3em}
\\

\hspace{-1em}{\bf Competent:} & HTML,
CSS,
Wordpress,
Linux,
Windows %\\
%{\bf Some Experience:} & C++, Parallel Computing, Mathematica, MATLAB, Javascript, Apple Script, ROOT, Swift\\
\end{tabular}
% {\bf Some Experience:} LabVIEW, Apple Script, Julia, Parallel Computing, Swift






% \section{Professional Societies}

% {\bf American Astronomical Society (AAS)}  \hfill {\bf 2014--present}
% \vspace{1em}
% % \begin{itemize}\itemsep -2pt
% %     \item[] Full Member \hfill {\bf 2019--present}
% %     \item[] Junior Member \hfill {\bf 2014--2018}
% % \end{itemize} \vspace{-10pt}
% % \textbf{\href{http://network.asa3.org}{American Scientific % Affiliation (ASA)}}, Student Member \hfill \textbf{2014 - % present}\\
% % {\bf American Physical Society (APS)}, Student Member  \hfill {\bf 2011--2014}








% \section{Publications in Preparation}

% \hangindent=15pt 
% {\bf Rose, B. M.}, Rubin, D., Dixon, S., et al., {\sl Testing Linear Standardization of Type Ia Supernovae using Gaussian Processes}





%url to copy and paste - https://ui.adsabs.harvard.edu/search/filter_property_fq_property=AND&filter_property_fq_property=property%3A%22refereed%22&fq=%7B!type%3Daqp%20v%3D%24fq_property%7D&fq_property=(property%3A%22refereed%22)&p_=0&q=orcid%3A0000-0002-1873-8973&sort=citation_count%20desc%2C%20bibcode%20desc
%All ORCID Papers -- https://ui.adsabs.harvard.edu/\#search/q=orcid\%3A\%220000-0002-1873-8973\%22&sort=date\%20desc\%2C\%20bibcode\%20desc
%Only Refereed papers -- https://ui.adsabs.harvard.edu/search/filter_property_fq_property=AND&filter_property_fq_property=property\%3A\%22refereed\%22&fq=\%7B!type\%3Daqp\%20v\%3D\%24fq_property\%7D&fq_property=(property\%3A\%22refereed\%22)&q=orcid\%3A0000-0002-1873-8973&sort=date\%20desc\%2C\%20bibcode\%20desc&p_=0
%Google Scholar -- https://scholar.google.com/citations?hl=en&user=7mgK134AAAAJ&view_op=list_works&gmla=AJsN-F6iePLN4Wjqx0zevRVxlzTXpDc-4tgHh3F1l7HgQKDk5_zZb36GX5SIz-2kA3vZQ80qinI8RGivZtpLqS9oI6iUuXFfSbvz_aHJrcP1iCaBmNpgHB0
%todo(should this link to newest on top, or most cited on top?)
\section{
\href{https://ui.adsabs.harvard.edu/search/filter_property_fq_property=AND&filter_property_fq_property=property\%3A\%22refereed\%22&fq=\%7B!type\%3Daqp\%20v\%3D\%24fq_property\%7D&fq_property=(property\%3A\%22refereed\%22)&p_=0&q=orcid\%3A0000-0002-1873-8973&sort=citation_count\%20desc\%2C\%20bibcode\%20desc}
{Refereed Publications\\  
\normalfont \textit{\small{\hspace{-0.3em}
% 4 first-author papers \\
% h-index = 8\\
1,033 total citations
% \\\dagger~ $>25$ citations
% \\\star~ $>10$ citations
}}}
}
% \newcommand{\twofive}{\hspace{-0.75em}\dagger~}
% \newcommand{\ten}{\hspace{-0.75em}\star~}

% \begin{etaremune}%[align=left]

\hangindent=15pt 
\href{https://ui.adsabs.harvard.edu/abs/2022arXiv220609950R/abstract}{{\bf Rose, B. M.}, Popovic, B., Scolnic, D., Brout, D. {\sl Constraining R$_V$ Variation Using Highly Reddened Type Ia Supernovae from the Pantheon+ Sample}, Accepted in MNRAS, \\arXiv:2206.09950}
\vspace{-12pt}

\hangindent=15pt 
\href{https://ui.adsabs.harvard.edu/abs/2022arXiv220606928M/abstract}{Meldorf, C., Palmese, A.,  Brout, D., \textbf{et al.} 2022, {\sl The Dark Energy Survey Supernova Program results: Type Ia Supernova brightness correlates with host galaxy dust}, Submitted to MNRAS, arXiv:2206.06928}
\vspace{-12pt}

\hangindent=15pt 
\href{https://ui.adsabs.harvard.edu/abs/2022arXiv220512949J/abstract}{Joshi, B., Strolger, L.,  Ryan, R., \textbf{et al.} 2022, {\sl High-Precision Redshifts for Type Ia Supernovae with the Nancy Grace Roman Space Telescope P127 Prism}, Submitted to ApJ, arXiv:2205.12949}
\vspace{-12pt}

\hangindent=15pt 
\href{https://ui.adsabs.harvard.edu/abs/2022arXiv220413553W/abstract}{Wang, K., Scolnic, D., Troxel, M., \textbf{et al.} 2022, {\sl A Synthetic Roman Space Telescope High-Latitude Time-Domain Survey: Supernovae in the Deep Field}, Submitted to MNRAS, arXiv:2204.13553}
\vspace{-12pt}

\hangindent=15pt 
\href{https://ui.adsabs.harvard.edu/abs/2022arXiv220412060G/abstract}{Garnavich, P., Wood, C. M., Milne, P., \textbf{et al.} 2022, {\sl Connecting Infrared Surface Brightness Fluctuation Distances to Type Ia Supernova Hosts: Testing the Top Rung of the Distance Ladder}, submitted to ApJ, arXiv:2204.12060}
\vspace{-12pt}

\hangindent=15pt 
\href{https://ui.adsabs.harvard.edu/abs/2022arXiv220210480C/abstract}{Chen, R., Scolnic, D., Rozo, E., \textbf{et al.} 2022, {\sl Measuring Cosmological Parameters with Type Ia Supernovae in redMaGiC galaxies}, submitted to ApJ, 
 arXiv:2202.10480}
\vspace{-12pt}

\hangindent=15pt 
\href{https://ui.adsabs.harvard.edu/abs/2022arXiv220204077B/abstract}{Brout, D., Scolnic D., Popovic, B., \textbf{et al.} 2022, {\sl The Pantheon+ Analysis: Cosmological Constraints}, accepted in ApJ, 
 arXiv:2202.04077}
\vspace{-12pt}

\hangindent=15pt 
\href{https://ui.adsabs.harvard.edu/abs/2021arXiv211203864B/abstract}{Brout, D., Taylor G., Scolnic D., Wood, C. M., \textbf{Rose, B. M.}, et al. 2021, {\sl The Pantheon+ Analysis: SuperCal-Fragilistic Cross Calibration, Retrained SALT2 Light Curve Model, and Calibration Systematic Uncertainty}, submitted to ApJ, 
 arXiv:2112.03864}
\vspace{-12pt}

\hangindent=15pt 
\href{https://ui.adsabs.harvard.edu/abs/2021arXiv211203863S/abstract}{Scolnic, D., Brout, D., Carr, A., {\bf et al.} 2021, {\sl The Pantheon+ Type Ia Supernova Sample: The Full Dataset and Light-Curve Release}, accpeted in ApJ, 
 arXiv:2112.03863}
\vspace{-12pt}

\hangindent=15pt 
% {\footnotesize \textcolor{light-gray}{8}} 
% \dagger 
\href{}{The SDSS Collaboration 2021, {\sl The Seventeenth Data Release Of The Sloan Digital Sky Surveys}, ApJSS submitted,} \href{http://arxiv.org/abs/2112.02026}{arXiv:2112.02026}
\vspace{-12pt}

\hangindent=15pt 
\href{https://ui.adsabs.harvard.edu/abs/2021arXiv211003487P/abstract}{Peterson, E. R., Kenworthy, W. D.,  Scolnic, D., {\bf et al.} 2021, {\sl The Pantheon+ Analysis: Evaluating Peculiar Velocity Corrections in Cosmological Analyses with Nearby Type Ia Supernovae}, accepted to ApJ, 
 arXiv:2110.03487}
\vspace{-12pt}

\hangindent=15pt 
\href{https://ui.adsabs.harvard.edu/abs/2021MNRAS.tmp.1758W/abstract}{Wiseman, P., Sullivan, M., Smith, M., {\bf et al.} 2021, {\sl Rates and Delay Times of Type Ia Supernovae in the Dark Energy Survey}, MNRAS, 506, 3330} 
% arXiv:2105.11954
\vspace{-12pt}

% \hangindent=15pt 
% \href{}{{\bf Rose, B. M.} 2021, {\sl Host-Galaxy Corrections Are a Red Herring: Type Ia Supernovae Are Nonlinear}, submitted to PASP on March 12, 2021} %http://arxiv.org/abs/2012.01460
% \vspace{-12pt}

\hangindent=15pt 
\href{https://ui.adsabs.harvard.edu/abs/2020arXiv201201460R/abstract}{{\bf Rose, B. M.}, Rubin, D., Strolger, L., Garnavich P. M. 2021, {\sl Combined, Host Galaxy Mass and Local Stellar Age Improves Type Ia Supernovae Distances}, ApJ, 909, 28} %http://arxiv.org/abs/2012.01460
\vspace{-12pt}

\hangindent=15pt 
% {\footnotesize \textcolor{light-gray}{8}} 
% \star 
\href{https://ui.adsabs.harvard.edu/abs/2020ApJ...896L...4R/abstract}{{\bf Rose, B. M.}, Rubin, D., Cikota, A., et al. 2020, {\sl Evidence for Cosmic Acceleration is Robust to Observed Correlations Between Type Ia Supernova Luminosity and Stellar Age}, ApJL, 896, L4}
% in press accepted May 15, 2020, %\href{http://arxiv.org/abs/2002.12382}{arXiv:2002.12382}
\vspace{-12pt}

\hangindent=15pt 
% {\footnotesize \textcolor{light-gray}{8}} 
% \dagger 
\href{https://ui.adsabs.harvard.edu/abs/2020ApJS..249....3A/abstract}{The SDSS Collaboration 2020, {\sl The Sixteenth Data Release Of The Sloan Digital Sky Surveys}, ApJSS, 249, 3}
%, in press, accepted May 11th, 2020, \href{http://arxiv.org/abs/1912.02905}{arXiv:1912.02905}
\vspace{-12pt}


\hangindent=15pt
% {\footnotesize \textcolor{light-gray}{7}}
% \href{https://ui.adsabs.harvard.edu/}{
\href{https://ui.adsabs.harvard.edu/#abs/2019arXiv191209993R/abstract}{{\bf Rose, B. M.}, Dixon, S., Rubin, D., et al.  2020, {\sl Initial Evaluation of SNEMO2 and SNEMO7 Standardization Derived From Current Light Curves of Type Ia Supernovae}, ApJ, 890, 60} %\href{http://arxiv.org/abs/1912.09993}{arXiv:1912.09993}%}
\vspace{-12pt}

\hangindent=15pt 
% {\footnotesize 6} 
% \dagger 
\href{https://ui.adsabs.harvard.edu/#abs/2019arXiv190201433R/abstract}{{\bf Rose, B. M.}, Garnavich, P. M., Berg, M. A. 2019, {\sl Think Global, Act Local: The Effect of Environment on Hubble Residuals of Type Ia Supernovae}, ApJ, 874, 32}
\vspace{-12pt}

\hangindent=15pt 
% {\footnotesize 5}
% \hspace{-1.1em}\dagger~
% \dagger 
\href{https://ui.adsabs.harvard.edu/#abs/2018arXiv181202759A/abstract}{The SDSS Collaboration 2019, {\sl The Fifteenth Data Release Of The Sloan Digital Sky Surveys}, ApJSS, 240, 23}
\vspace{-12pt}

\hangindent=15pt 
% {\footnotesize 4}
% \star 
\href{https://ui.adsabs.harvard.edu/#abs/2016ApJ...827...60M/abstract}{Mathews, G. J., {\bf Rose, B. M.}, Garnavich, P. M., et al. 2016, {\sl Detectability of Cosmic Dark Flow in the Type Ia Supernova Redshift-Distance Relation}, ApJ, 827, 60}

\vspace{-12pt}
\hangindent=15pt 
% {\footnotesize 3}
\href{https://ui.adsabs.harvard.edu/#abs/2016AJ....152...27K/abstract}{Kennedy, M.~R., Callanan, P., Garnavich, P.~M., {\bf et al.} 2016, {\sl The New Eclipsing CV MASTER OTJ192328.22+612413.5: A Possible SW Sextantis Star}, AJ, 152, 27}

\vspace{-12pt}
\hangindent=15pt 
% {\footnotesize 2} 
\href{https://ui.adsabs.harvard.edu/#abs/2015IJMPA..3045022M/abstract}{Mathews, G.~J., Gangopadhyay, M.~R., Garnavich, P., {\bf Rose, B. M.}, et al. 2015, {\sl Constraints on the Birth of the Universe and Origin of Cosmic Dark Flow}, International Journal of Modern Physics A, 30, 1545022}

\vspace{-12pt}
\hangindent=15pt 
% {\footnotesize 1}
\href{https://ui.adsabs.harvard.edu/#abs/2015IBVS.6129....1L/abstract}{Littlefield, C., Garnavich, P., Magno, K., {\bf et al.} 2015, {\sl High-Amplitude, Rapid Photometric Variation of the New Polar MASTER OT J1321}, Information Bulletin on Variable Stars, 6129, 1}

\section{Roman\\Technical\\Reports}

\hangindent=15pt
\textbf{Rose, B. M.}, et al. 2021 {\sl A Reference Survey for Supernova Cosmology with the Nancy Grace Roman Space Telescope}, Report to Roman Project,
% September 1, 2021
\href{https://arxiv.org/abs/2111.03081}{arXiv:2111.03081}
\vspace{-12pt}

%Shared never posted
\hangindent=15pt 
\textbf{Rose, B. M.},  R. Hounsell,  S. Deustua,  et al. 2021, {\sl Prioritization of RCS LED Lenses: Impacts on the Supernova Key Project}, Memo for Roman Calibration Working Group, March 11, 2021
\vspace{-12pt}

\hangindent=15pt 
\href{iopscience.iop.org/article/10.3847/2515-5172/abf1fb}{Deustua, S., \textbf{et al.} 2021, {\sl The Roman Space Telescope Relative Calibration System and the Dark Energy Figure of Merit}, Res. Notes AAS 5 66}
\vspace{-12pt}

%Shared never posted
\hangindent=15pt 
\textbf{Rose, B. M.}, Rubin, D., Deustua, S., et al. 2020, {\sl The Limit of Pre-flight Unusable Pixels on the Roman Space Telescope Supernova Survey Science}, Roman Detector Working Group, August 28, 2020
\vspace{-12pt}

\hangindent=15pt 
Ryan, R. E., Crawford, S., \textbf{et al.} 2020, {\sl Anticipated Data Processing and Algorithm Descriptions for SIT-Contributed Software.} \href{https://outerspace.stsci.edu/display/FWG/Anticipated+Data+Processing+and+Algorithm+Descriptions+for+SIT-Contributed+Software}{A WFIRST operations concept document}


\section{\href{https://ui.adsabs.harvard.edu/\#search/q=orcid\%3A\%220000-0002-1873-8973\%22&sort=date\%20desc\%2C\%20bibcode\%20desc}{Non-refereed Works}}

\hangindent=15pt 
\textbf{Rose, B. M.}, et al. 2021, {\sl Synergies between Vera C. Rubin Observatory, Nancy Grace Roman Space Telescope, and Euclid Mission: Constraining Dark Energy with Type Ia Supernovae}, Response to DOE/NASA request for information on January 21, 2021. \href{https://arxiv.org/abs/2104.01199}{arXiv:2104.01199}
\vspace{-12pt}

%Written never shared
% \hangindent=15pt \href{}{{\bf Rose, B. M.} 2019, {\sl Supernovae Survey Cadence and the Ability to Constrain Type Ia Systematic Uncertainties}, WFRIST Note, arXiv:0000.000000, \textit{in prep.}}
% \vspace{-12pt}

\textit{Endorsed Astro2020 Science Whitepapers:}

\vspace{-12pt}\hspace{15pt}\hangindent=30pt 
Keith, B., et al. \textit{Dark Matter Science in the Era of LSST}
\href{https://ui.adsabs.harvard.edu/abs/2019arXiv190304425B/abstract}{arXiv:1903.04425}

\vspace{-12pt}\hspace{15pt}\hangindent=30pt 
Green, D., et al. \textit{Messengers from the Early Universe: Cosmic Neutrinos and Other Light Relics}
\href{https://ui.adsabs.harvard.edu/abs/2019arXiv190304763G/abstract}{arXiv:1903.04763}

\vspace{-12pt}\hspace{15pt}\hangindent=30pt 
Foley, R., et al. \textit{WFIRST: Enhancing Transient Science and Multi-Messenger Astronomy}
\href{https://ui.adsabs.harvard.edu/abs/2019arXiv190304582F/abstract}{arXiv:1903.04582}

\vspace{-12pt}\hspace{15pt}\hangindent=30pt
Mantz, A., et al. \textit{The Future Landscape of High-Redshift Galaxy Cluster Science}\\
\href{https://ui.adsabs.harvard.edu/abs/2019BAAS...51c.279M/abstract}{arXiv:1903.05606}

\vspace{-12pt}\hspace{15pt}\hangindent=30pt 
Williams, B., et al. \textit{Far Reaching Science with Resolved Stellar Populations in the 2020s}
\href{https://ui.adsabs.harvard.edu/abs/2019BAAS...51c.301W/abstract}{BAAS51c.301W}

\vspace{-12pt}\hspace{15pt}\hangindent=30pt 
Sehgal, N., et al. \textit{Science from an Ultra-Deep, High-Resolution Millimeter-Wave Survey}
\href{https://ui.adsabs.harvard.edu/abs/2019arXiv190303263S/abstract}{arXiv:1903.03263}

\vspace{-12pt}\hspace{15pt}\hangindent=30pt 
Dore, O., et al. \textit{WFIRST: The Essential Cosmology Space Observatory for the Coming Decade}
\href{https://ui.adsabs.harvard.edu/abs/2019arXiv190401174D/abstract}{arXiv:1904.01174}

\vspace{-12pt}
\hangindent=15pt \href{https://ui.adsabs.harvard.edu/abs/2014GCN..16492...1G/abstract}{Garnavich, P. \& {\bf Rose, B. M.} 2014, {\sl GRB140629A: VATT optical observations}, GRB Coordinates Network, Circular Service, No. 16492, \#1}




\section{Invited \\Presentations}

\hangindent=15pt 
% {\footnotesize 6}
\textit{Taking Supernova Cosmology from DES to Roman}, April 20, 2022, U. Pennsylvania Astrophysics Semianr, Philadelphia, Pennsylvania
\vspace{-12pt}

\hangindent=15pt 
% {\footnotesize 6}
%https://youtu.be/f1TT4HEJspA
%https://conference.ipac.caltech.edu/romantimedomain/pdfs/session3_Rose.pdf
%https://conference.ipac.caltech.edu/romantimedomain/page/agenda
\textit{Synergies of the Roman Space Telescope with Other Missions and Facilities}, February 8, 2022, co-led a panel discussion at Exploring the Transient Universe with the Nancy Grace Roman Space Telescope, Pasadena, California
\vspace{-12pt}

\hangindent=15pt 
% {\footnotesize 6} 
{\it Updates from Roman Supernova Science Investigation Team}, May 19, 2021, LSST DESC Seminar, virtual
\vspace{-12pt}

\hangindent=15pt 
% {\footnotesize 6} 
{\it Systematics Limited Cosmology with the Nancy Grace Roman Space Telescope}, September 22, 2020, Notre Dame Astrophysics Seminar, virtual
\vspace{-12pt}

\hangindent=15pt 
% {\footnotesize 6} 
{\it WFRIST and Type Ia Supernova Systematic Uncertainties}, April 21, 2020 (\textit{Postponed}), Notre Dame Astrophysics Seminar, South Bend, Indiana
\vspace{-12pt}

\hangindent=15pt 
% {\footnotesize 6} 
{\it Understaning Type Ia Supernova Systematic Uncertainties}, April 9, 2020 (\textit{Canceled}), Duke Astrophysics Seminar, Durham, North Carolina




\section{Oral \\Presentations}

\hangindent=15pt 
{\sl A Forecast of Extragalactic Transient Light Curves for the Roman Time Domain Core Community Survey}, August 2022, Time Domain and Multi-Messenger Astrophysics NASA Workshop, Annapolis, Maryland
\vspace{-12pt}

\hangindent=15pt 
{\sl Constraining $R_V$ Variation Using Highly Reddened Type Ia Supernovae from the Pantheon+ Sample}, June 16, 2022, 240th AAS Meeting, Pasadena, California
\vspace{-12pt}

\hangindent=15pt 
% {\footnotesize 8} 
{\sl A High-Latitude Time Domain Reference Survey}, November 18, 2021, Roman Science Team Community Briefing, \href{https://roman.gsfc.nasa.gov/science/workshop112021/agenda.html#sn}{virtual}
\vspace{-12pt}


% \hangindent=15pt 
% % {\footnotesize 8} 
% {\sl Improving SN Ia Standardization with Host Galaxy Mass and Local Stellar Age}, January 16, 2021, DESC Supernova Working Group meeting
% \vspace{-12pt}

\hangindent=15pt 
% {\footnotesize 8} 
{\sl Improving SN Ia Standardization with Host Galaxy Mass and Local Stellar Age}, January 11, 2021, 237th AAS Meeting, virtual
\vspace{-12pt}

\hangindent=15pt 
% {\footnotesize 7} 
{\sl Potential Systemics from Standardizing Our Standard Candles}, March 2, 2020, \href{https://indico.flatironinstitute.org/event/122/#43-potential-systemics-from-st}{WFIRST Science Jamboree, New York, New York}
\vspace{-12pt}

% science coffee (SENMO/UNITY) at stsci, 16 October 2019

\hangindent=15pt 
% {\footnotesize 6} 
{\sl Initial Evaluation of SNEMO2 and SNEMO7 Standardization}, October 3, 2019, SNIa Cosmology Analysis Meeting, Chicago, Illinois
\vspace{-12pt}

\hangindent=15pt 
% {\footnotesize 5} 
{\sl Tools for Supernova Standardization: 
Bayesian Hierarchical Models}, July 30th, 2019,
WFIRST Science Jamboree, Greenbelt, Maryland
\vspace{-12pt}


% Townson high talk, on  June 5th 2019


\hangindent=15pt 
% {\footnotesize 4} 
Dissertation Talk. {\sl Think Local, Act Global: The Influence of Host Galaxy Properties on Type Ia Light Curves}, January 9th, 2019,
233rd AAS Meeting, Seattle, Washington
\vspace{-12pt}

% science coffee (Rose++ 2019a) at stsci, 30 November 2018

\hangindent=15pt 
% {\footnotesize 3} 
{\sl Searching For a Cosmic-scale Dark Flow}, November 20, 2015,
2015 APS Prairie Section Meeting, Notre Dame, Indiana
\vspace{-12pt}

\hangindent=15pt 
% {\footnotesize 2} 
{\sl Finding A Cosmic Bulk Flow}, April 28, 2014, 2014 GPS Spring Conference, Notre Dame, Indiana
\vspace{-12pt}

\hangindent=15pt 
% {\footnotesize 1} 
{\sl Determining the Location of a Radioactive Source in \textsc{Majorana Demonstrator}}, August 2, 2011, REU Culminating Talks, Duke University, Durham, North Carolina

\begin{comment}
{\sl Searching For a Cosmic-scale Dark Flow} \hfill {\bf November 20, 2015}\\
2015 APS Prairie Section Meeting, Notre Dame

{\sl Finding A Cosmic Bulk Flow} \hfill {\bf April 28, 2014}\\
2014 GPS Spring Conference, Notre Dame 

{\sl Determining the Location of a Radioactive Source in \textsc{Majorana}} \hfill {\bf August 2, 2011} \\
{\sl\textsc{Demonstrator}} \\
REU Culminating Talks,
Duke University %\hfill {\bf August 2, 2011} 
\end{comment}




%  $_{•}    $\\\centerline{\bf Talks:} \\ \\
% {\bf Searching For a Cosmic-scale Dark Flow \hfill November 20, 2015} \\
% {\sl 2015 APS Prairie Section Meeting} \\
% The University of Notre Dame \\
% South Bend, IN

% {\bf Finding A Cosmic Bulk Flow \hfill April 28, 2014} \\
% {\sl 2014 GPS Spring Conference} \\
% The University of Notre Dame \\
% South Bend, IN

% {\sl Galaxies talk 3}  \\
% The University of Notre Dame \\
% South Bend, IN
% \begin{itemize}\itemsep -2pt
%      \item A class presentation given as an author on a significant paper from the literature
% \end{itemize}

% {\sl Galaxies talk 2}  \\
% The University of Notre Dame \\
% South Bend, IN
% \begin{itemize}\itemsep -2pt
%      \item A class presentation given as an author on a significant paper from the literature
% \end{itemize}

% {\bf Buldeless Dwarf Galaxies, Presenting on Govenato et al. 2010 \\
%  $_{•}    $ \hfill February 14, 2014}  \\
% The University of Notre Dame \\
% South Bend, IN
% \begin{itemize}\itemsep -2pt
%      \item A class presentation given as an author on a significant paper from the literature
% \end{itemize}

% {\bf Determining the Location of a Radioactive Source in \textsc{Majorana Demonstrator} \\
% $_{•}    $  \hfill August 2, 2011} \\
% {\sl Research Experience for Undergraduates Culminating Talks}  \\
% Duke University \\
% Durham, NC
% \begin{itemize}\itemsep -2pt
%      \item APS style presentation of summer research
% \end{itemize}




\section{Poster Presentations}
%  $_{•}    $\\
% \centerline{\bf Posters:}
\hangindent=15pt 
% {\footnotesize 9} 
{\sl Constraining $R_V$ Variation Using Highly Reddened Type Ia Supernovae from the Pantheon+ Sample}, January 2022, 239th AAS Meeting, Salt Lake City, Utah, \textit{canceled}
\vspace{-12pt}


\hangindent=15pt 
% {\footnotesize 9} 
\href{https://aas236-aas.ipostersessions.com/default.aspx?s=02-A5-6C-28-27-98-74-49-ED-38-21-BF-C2-85-DA-39}{\sl Testing Linear Standardization of Type Ia Supernovae using Gaussian Processes}, June 2020, 236th AAS Meeting, virtual
\vspace{-12pt}

\hangindent=15pt 
% {\footnotesize 9} 
{\sl Can Type Ia Supernovae Systematics Resolve the Current Hubble Tension?}, January 2020, 235th AAS Meeting, Honolulu, Hawaii
\vspace{-12pt}

\hangindent=15pt 
% {\footnotesize 8} 
{\sl Estimating the Average Age of Stellar Populations to Understand Type Ia Supernova Systematics}, November 18, 2019, The Art Of Measuring Galaxy Physical Properties, Milan, Italy
\vspace{-12pt}
\vspace{-12pt}

\hangindent=15pt 
% {\footnotesize 7}
{\sl Can Type Ia Supernovae Systematics Resolve the Current Hubble Tension?}, October 5, 2019, Cosmic Controversies, Chicago, Illinois
\vspace{-12pt}

\hangindent=15pt 
% {\footnotesize 6} 
\href{https://ui.adsabs.harvard.edu/#abs/2018AAS...23124512R/abstract}{{\sl Correlations Between Hubble Residuals and MCMC Estimated Local Stellar Ages of Type Ia Supernovae}, January 10, 2018}, 231th AAS Meeting, Washington, DC
\vspace{-12pt}

% repeat of AAS 229th at GPSAC 2017 on September 13th at ND. http://gradphysics.nd.edu/conference/gpsac-2017/

\hangindent=15pt 
% {\footnotesize 5} 
\href{https://ui.adsabs.harvard.edu/#abs/2017AAS...22943402R/abstract}{{\sl Correlations Between Hubble Residuals and Local Stellar Populations of Type Ia Supernovae}, January 7, 2017}, 229th AAS Meeting, Grapevine, Texas
\vspace{-12pt}

\hangindent=15pt 
% {\footnotesize 4} 
\href{https://ui.adsabs.harvard.edu/#abs/2016AAS...22723711R/abstract}{{\sl Correlating Type Ia Supernova Properties With Their Local Environment Using HST Snapshots of Host Galaxies}}, January 6, 2016, 227th AAS Meeting, Kissimmee, Florida
\vspace{-12pt}

\hangindent=15pt 
% {\footnotesize 3}
{\sl Prospects for Detecting a Cosmic Bulk Flow},  January 6, 2015, 225th AAS Meeting, Seattle, Washington 
\vspace{-12pt}

\hangindent=15pt 
% {\footnotesize 2} 
{\sl Finding A Cosmic Bulk Flow}, February 27, 2014, GSU 6th Annual Research Symposium, Notre Dame, Indiana
\vspace{-12pt}

\hangindent=15pt 
% {\footnotesize 1} 
{\sl Determining the Location of a Radioactive Source in \textsc{Majorana Demonstrator}}, October 27, 2011, APS, Division of Nuclear Physics, Michigan State University 


\begin{comment}
\href{https://ui.adsabs.harvard.edu/#abs/2017AAS...22943402R/abstract}{{\sl Correlations Between Hubble Residuals and Local Stellar Populations}} \hfill {\bf January 7, 2017} \\
\textit{of Type Ia Supernovae}\\
AAS 229th Meeting, Grapevine, Texas

\href{https://ui.adsabs.harvard.edu/#abs/2016AAS...22723711R/abstract}{{\sl Correlating Type Ia Supernova Properties With Their Local}} \hfill {\bf January 6, 2016} \\
\textit{Environment Using HST Snapshots of Host Galaxies} \\
AAS 227th Meeting, Kissimmee, Florida

{\sl Prospects for Detecting a Cosmic Bulk Flow} \hfill {\bf January 6, 2015}\\
AAS 225th Meeting, Seattle, Washington  

{\sl Finding A Cosmic Bulk Flow}\hfill {\bf February 27, 2014}\\
GSU 6th Annual Research Symposium, Notre Dame 

{\sl Determining the Location of a Radioactive Source in \textsc{Majorana}} \hfill {\bf October 27, 2011}\\
\textit{\textsc{Demonstrator}} \\
APS, Division of Nuclear Physics, Michigan State University 
% $_{•}    $ \hfill {\bf October 27, 2011}

% {\sl Correlating Type Ia Supernova Properties with Their Local Environment Using HST Snapshots of Host Galaxies}. {\bf Rose, B.}, \& Garnavich, P. AAS 227th Meeting, Kissimmee, FL. January 6, 2016

% {\sl Prospects for Detecting a Cosmic Bulk Flow}. {\bf Rose, B.}, Garnavich, P., Mathews, G. J. AAS 225th Meeting, Seattle, WA. January 6, 2015

% {\sl Finding A Cosmic Bulk Flow}. {\bf Rose, B.}, Garnavich, P., Mathews, G. J. Graduate Student Union 6th Annual Research Symposium, University of Notre Dame, South Bend, IN. February 27, 2014

% {\sl Determining the Location of a Radioactive Source in \textsc{Majorana Demonstrator}} {\bf Rose, B.} APS, Division of Nuclear Physics. Michigan State University, East Lansing, MI. October 27, 2011

% American Physical Society, Department of Nuclear Physics\\ 
% Michigan State University \\
% East Lansing, MI 

% Old presentation method - like Talks, currently prefer posters to be like papers
% {\bf Correlating Type Ia Supernova Properties with Their Local Environment Using HST Snapshots of Host Galaxies  \hfill January 6, 2016} \\
% {\sl AAS 227th Meeting} \\
% Kissimmee, FL

% {\bf Prospects for Detecting a Cosmic Bulk Flow \hfill January 6, 2015} \\
% {\sl AAS 225th Meeting} \\
% Seattle, WA

% {\bf Finding A Cosmic Bulk Flow  \hfill February 27, 2014} \\
% {\sl Graduate Student Union 6th Annual Research Symposium} \\
% The University of Notre Dame \\
% South Bend, IN

% % \subsubsection{Poster Sessions:}
% % {\sl Finding A Cosmic Bulk Flow} \\
% % event Name \hfill February ??, 2014\\
% % University of Notre Dame \\
% % South Bend, IN

% \href{<http://meeting.aps.org/Meeting/DNP11/Session/EA.115>}{\bf Determining the Location of a Radioactive Source in \textsc{Majorana Demonstrator}} \\
% {\bf $_{•}    $  \hfill October 27, 2011 }\\
% American Physical Society, Department of Nuclear Physics\\ 
% Michigan State University \\
% East Lansing, MI 
\end{comment}














% change this to outreach(/leadership maybe) & research interests.

% \section{Lab Work\\ \& \\Projects}
% {\sl Searching for a Cosmic Bulk Flow} \hfill ?? 
% \begin{itemize}\itemsep -2pt
%      \item Searching for a Dark Flow signal using supernovae (SNe)
%      \item Working with multiple SNe catalogs
%      \item Developing my own analysis code
% \end{itemize}

% {\sl Physics Outreach} \hfill Spring 2012 
% \begin{itemize}\itemsep -2pt
% 	\item Worked with middle school students on high altitude weather balloon experiments
% 	\item Personally worked on radiation and ozone detector systems, as well as temperature sensors
% \end{itemize}

% {\sl Determining the Location of a Radioactive Source in \textsc{Majorana Demonstrator}}
%   \\ $_{•}    $                   \hfill Summer 2011
% \begin{itemize}\itemsep -2pt
% 	\item Presented at poster session of APS DNP 2011 
% 	\item Worked at TUNL analyzing Monte Carlo generated data to determine how well  \textsc{Majorana} can resolve the location of a radioactive hot spot
% \end{itemize}

% {\sl Cosmic Rays} \hfill Fall 2010 
% \begin{itemize}\itemsep -2pt
% 	\item Detected Cosmic Rays with respect to altitude using a three Geiger counter telescope
% 	\item Used a high altitude weather balloon
% \end{itemize}

% {\sl Electronic Robot} \hfill Spring 2010 
% \begin{itemize}\itemsep -2pt
% 	\item Electronics class final project
% 	\item Designed and built a robot that wrote and erased on a whiteboard
% \end{itemize}

% {\sl Differential Equations} \hfill Fall 2009
% \begin{itemize}\itemsep -2pt
% 	\item Differential Equations class final project
% 	\item Mapping the movement of a building during an earthquake by using numerical methods in MATLAB 
% \end{itemize}


\end{resume}
\end{document}






%%%%%%%%%%%%%%%%%%%%%%%%%%%%%%%%%%%%%%%%%%%%%%%%%%%%%%%%%%%%%%%%%%%%%%%%%%%%%%%%%%%%%%%%%%%%
%%%%%%%%%%%%%%%%%%%%%%%%%%%%%%%%%%%%%%%%%%%%%%%%%%%%%%%%%%%%%%%%%%%%%%%%%%%%%%%%%%%%%%%%%%%%
%%%%%%%%%%%%%%%%%%%%%%%%%%%%%%%%%%%%%%%%%%%%%%%%%%%%%%%%%%%%%%%%%%%%%%%%%%%%%%%%%%%%%%%%%%%%
\begin{comment}
\entry{Research Assistant \hfill January 2014 - Present\\}
{\bf Research Assistant \hfill January 2014 - Present}
The University of Notre Dame 
     
{\sl Teacher Assistant} \hfill August 2012 - December 2013 \\
The University of Notre Dame 
\begin{itemize}\itemsep -2pt
\item More detail under Teaching Experience.
\end{itemize}

{\sl Supplemental Instruction (SI) Leader for Physics} \hfill September 2011 - May 2012 \\
 Whitworth University
\begin{itemize}  \itemsep -2pt %reduce space between items
     \item Led a    group study session of introductory physics material
     \item Implemented active learning techniques
\end{itemize}

{\sl Research Experience for Undergraduates} \hfill Summer 2011 \\
Triangle Universities Nuclear Laboratory (TUNL) \\ 
Durham, NC
\begin{itemize}  \itemsep -2pt %reduce space between items
     \item Worked on \textit{Determining the Location of a Radioactive Source in \textsc{Majorana Demonstrator}} described in ``Lab Work \& Projects" 
\end{itemize}

{\sl Laboratory Teaching Assistant} \hfill February 2011 - May 2011 \\
 Whitworth University
\begin{itemize}  \itemsep -2pt %reduce space between items 
     \item Assisted in and graded an introductory physics lab session
\end{itemize}

{\sl Personal Tutor} \hfill September 2010 - May 2011 \\
 High School Student, Algebra II and Geometry
\begin{itemize}  \itemsep -2pt %reduce space between items
     \item One-on-one sessions, reinforcing content and working on homework
\end{itemize}

{\sl Daycare Teacher} \hfill January 2009 - May 2011 \\
 Whitworth Community Presbyterian Church
\begin{itemize}  \itemsep -2pt %reduce space between items
     \item Responsible for up to 10 preschoolers, worked as a part of a team, planned kids activities, part of closing team, created a safe environment, encouraged creativity, and reinforced developmental skills
\end{itemize}

{\em University of Notre Dame:}
\begin{itemize} \itemsep -2pt %reduce space between items
     \item Laboratory Teaching Assistant and Grader
     \begin{itemize} \itemsep -2pt %reduce space between items
          \item Physics 1 for Life Science \hfill Fall 2013, Summer 2013
          \item Phycics 2 for Engineers \hfill Spring 2013
          \item Physics 2 for Life Science \hfill Fall 2012
     \end{itemize}
     \item Exam Grader \hfill Fall 2012 - Fall 2013
\end{itemize}
{\em Whitworth University:}
\begin{itemize} \itemsep -2pt %reduce space between items
     \item SI Instructor
     \item Laboratory Teaching Assistant and Grader
\end{itemize}

personal tutor:One-on-one sessions, reinforcing content and working on homework
\end{comment}